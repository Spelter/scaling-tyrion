% !TEX root=../document.tex

\section{Stakeholder Descriptions}

ReVision is being developed in co-operation with Revolve NTNU
 \cite{kranebae:fordypning}, and the specifications and requirements for the
system are partially given by the frames set by the Formula Student (FS) 
competition-rules \cite{FS:challenge,FSG:disciplines,fsae:2014rules}.
The competition rules dictate that there exists a business plan for the
developed car, targeting the car towards `the nonprofessional weekend autocross racer'
 \cite{FSG:disciplines}. 

Considering a working business model would allow the product to
be sold to consumers or enterprises that provides racing experiences to
consumers these should also be considered as stakeholders. This should result
in more realistic requirements.

\subsection{Stakeholder (Non-User) Summary}

Two obvious stakeholders are Revolve NTNU and the judges in the FS
competitions. Also any enterprise owning vehicles with such systems are
non-user stakeholders. Common for all of these stakeholders is that the
attractiveness of the system is important. For Revolve NTNU the system could
help attain higher scores in the competition, both in the static and dynamic
events \cite{FSG:disciplines}. For the judges, who in the competitions
pretend to be investors, the system's ability to differentiate itself from the
competition in a positive way is important, but also the possibility that such a system could
attract customers is something that should result in a higher score.

Another stakeholder are enterprises purchasing the system for integration into existing vehicles,
or purchasing vehicles with the system integrated. In this case the ability to
differentiate itself from other providers of similar services is important, so
is the ability to customize the product and experience to suit customer needs
and demand.

\subsection{User Summary}
The users of the system consists of two primary groups; professionals and
non-professionals. Common for both of them is that they will use to system for
training purposes. Professionals will also utilize the system in race contexts,
and the non-professionals might do so.

\subsection{Key High-Level Goals and Problems of the Stakeholders}


% TODO Make smaller margins around the 
%nasty way of wrapping table acroos multiple pageslists in the table.
\clearpage
\begin{table}\small
\begin{tabularx}{\textwidth}{|X|l|X|X|}
	\hline
	\rowcolor{gray!50}
	High-Level Goal & Priority & Problems and Concerns & Current Solutions \\
	\hline
	Real-time processing and visualization of acquired data 
		& High 
			& More time needed for processing as amount of data grows. \newline \newline More time needed for processing as graphical transformations. 
				
				& Existing systems avoids these problems by being simple and handle less
				complex information. \\
	\hline
	Flexible architecture allowing portability and adaptability
		& High
			& Too tailored architecture to one design will make it difficult to
				reuse the system, or create a new implementation for another design. \newline \newline Lack of configuration flexibility will increase testing time and leave
				little room for further innovation.
				& Existing systems do not address these issues. \\
	\hline
	Ability to accumulate data and/or display certain data at triggered points,
	and the ability to define these triggers.
		& Medium
			& Certain events, timed occurrences can trigger the need to show data (lap time for instance). \newline \newline Giving more information can be disturbing to the driver under certain circumstances.
				& Existing systems output information when it is received/registered and
				therefore do not address this issue.\\
	\hline
	Robust and reliable data processing
		& High
			& Data loss or corruption can mislead or distract driver if not handled correctly. \newline \newline Wireless communication increases chances of data loss and failures.
				& Existing systems doesn't indicate anything when no information is
				received. Robustness is given through wired connections. \\
	\hline
\end{tabularx}
\end{table}
%nasty way of wrapping table acroos multiple pages
\begin{table}\small
\begin{tabularx}{\textwidth}{|X|l|X|X|}
        \hline
	General support for networking standards
		& Low
			& Hard-coded support for networking makes it difficult to move the system
			over to a design utilizing another network standard
				& Existing systems are usually hard-coded to the network used in the
				design. \\
	\hline
\end{tabularx}
\end{table}
\clearpage

\subsection{User-Level Goals}
The users need a system to fullfill these goals:
\begin{itemize}
	\item Developer: Configure message formats, Configure sensors, Configure
	display, Upload configuration
	\item System implementer: Configure message formats, Configure sensors, Configure
	display, Upload configuration
	\item System owner: Configure sensors, Configure
	display, Upload configuration, Display real-time information, Receive and display messages
	\item Revolve NTNU: Configure message formats, Configure sensors, Configure
	display, Upload configuration, Display real-time information, Receive and display messages
	\item Driver: Configure
	display, Upload configuration, Display real-time information, Receive and display messages
\end{itemize}

\subsection{User Environment}
The user environment is in the driver seat of a race car. For Revolve NTNU
this is an one-seater open formula student class race car, but theoretically it
could be any kind of vehicle construction. The system is designed to be mounted
or worn on the inside of the helmet, with minimal or no interaction from the
driver during use/race conditions. Interaction with the system is mainly done
from a secondary control system.
