%Abbreviations and definitions
\label{sec:abbriv}
\vspace{5mm}

\begin{description}
\item[IMechE] Institution of Mechanical Engineers
\item[ICE] Internal Combustion Engine
\item[CSMA/CD] Carrier Sense Multiple Access with Collision Detection
\item[CAN] Controller Area Network, CSMA/CD network standard for automotive applications.
\item[HUD] Heads-up display. Display where visual information is within or in
close proximity to the viewers normal field-of-view.
\item[HDD] Heads-down display. Display where visual information is outside the
viewers normal field-of-view.
\item[SuD] System under Discussion.
\item[DIS] \label{abbriv:DIS}Driver Information System. A computerized
information system thats purpose is to provide real-time information to the
driver of critical data; speed, fuel, coolant temperature, and warnings like
oil pressure and brake system. 
\item[IVIS] In Vehicle Information System, see DIS
\vspace{10mm}
\item[Black-box] A way of describing a technical system that implies no
knowledge of the systems internal design. It only shows the systems interaction
with other systems.
\item[Sensor] Device that takes a physical input and gives a mechanical or
electrical output. Devices range from simple mechanical switches (toggles
electrical current on/off) to complex semiconductor based components reacting
to physical forces like magnetism and pressure, detects metal and reacts to
changes temperature.
\item[Instrument] Device used to indicate states of a system. Can use simple
devices like lamps, speakers, and gauges, or complicated devices like screens
and vibration-motors. 
\item[Gauge] In the context of instrumentation: Device that typically uses a dial to show a value of some sort.
Common examples are speed, fuel and temperature gauges.
\end{description}


