\section{Mathematical Model}
	\label{sec:model}

Citing in latex: The mathematical foundation for this case is available in the project thesis by Øgård. \cite{project} 

\noindent Here are some equations for example.

\subsection{Pressure force contribution}
By continuing the work started there, and using the expression for the pressure gradient, an expression for the pressure can be found.

	\begin{multline}
		P(\phi,t) - P(-\frac{\pi}{2}) = \int_{-\frac{\pi}{2}}^{\phi} \frac{\partial p}{\partial \phi} d\phi = \\
- \mu u_y(t) \int_{-\frac{\pi}{2}}^{\phi} \left( \frac{r_1}{r_2 - r_i(\phi,t)} + K_{\beta}(\phi,t)\right)\frac{cos(\phi)}{K_{\alpha}(\phi,t)}d\phi
	\label{eq:pressure_distribution}
	\end{multline}\\

\noindent The pressure difference is of interest and for that reason the reference pressure $P(-\frac{\pi}{2})$ is set to zero for all values of t. The pressure distribution along the clearance can be used to find the active force counteracting the movement of the inner cylinder. 

For symmetrical reasons the only component of the pressure that counteracts the inner cylinder movement is the y-component. The pressure force per length unit is then given by:

	\begin{equation}
		\frac{F_{p,y}(t)}{L} = -2 \int_{-\frac{\pi}{2}}^{\frac{\pi}{2}}P(\phi,t)r_1 \; sin(\phi) d\phi = -\mu u_y(t) C_p(t)
	\label{eq:pressure_force}
	\end{equation}\\

\noindent Where the constant $C_p$ is defined as:

	\begin{equation}
		C_p(t) = 2 \int_{-\frac{\pi}{2}}^{\frac{\pi}{2}}\left[ \int_{-\frac{\pi}{2}}^{\phi} \left( \frac{\frac{r_1}{r_i}}{(\gamma-1)} + K_{\beta} \right)\frac{-r_1}{K_{\alpha}} cos(\phi) d\phi \right] sin(\phi) d\phi
	\label{eq:pressure_force_const}
	\end{equation}\\

\noindent This result need to be compared with the simulated values for different geometries. \Vref{eq:pressure_force} may be solved numerically in order to get a time-dependent solution for a given case.

\subsection{Viscous force contribution}

In order to find the viscous force contribution on the inner cylinder the expression for the viscous stress in cylindrical coordinates needs to be used.

	\begin{equation}
		\tau_{r \phi} = -\mu r \frac{\partial}{\partial r} \left(\frac{u_{\theta}}{r} \right) \bigg|_{r=r_i} = 2 \mu \frac{B}{R_i^2}-\frac{1}{2}\frac{\partial p}{\partial \phi}
	\label{eq:Viscous_distribution}
	\end{equation}\\

\noindent The same symmetry argument as for the pressure force applies for the viscous force. The y-component of the viscous force per length unit is:

	\begin{equation}
		\frac{F_{s,y}(t)}{L} = -2 \int_{-\frac{\pi}{2}}^{\frac{\pi}{2}}r_i \tau(\phi,t) \; cos(\phi) d\phi = -\mu u_y(t) C_s(t)
	\label{eq:viscous_force}
	\end{equation}\\

\noindent Where the constant $C_s$ is defined in the following way:

	\begin{equation}
		C_s(t) = 2 \int_{-\frac{\pi}{2}}^{\frac{\pi}{2}}\left[ -\frac{r_1}{\alpha} \left( \frac{\frac{r_1}{r_i}}{(\gamma-1)} + K_{\beta} \right) \right]
	\label{eq:viscous_force_const}
	\end{equation}\\

\noindent The structure of the viscous force is of similar structure as the pressure force and the only variation is in the variables $C_s \& C_p$. The values for $K_{\beta}, K_{\alpha}, \gamma \& r_i$ is defined in the following way.

\begin{equation}
r_i = \sqrt{r_1^2 + \Delta y^2(t) + 2r_1 \Delta y(t) sin(\theta)}
\label{eq:const_ri}
\end{equation}\\

\begin{equation}
\gamma = \frac{r_2}{r_i}
\label{eq:const_gamma}
\end{equation}\\

\begin{multline}
K_{\alpha} = \frac{1}{4(r_2 - r_i)} \left[ \frac{2r_i^2 r_2^2}{r_2^2 - r_i^2}ln^2(\frac{r_2}{r_i}) - \frac{r_2^2 - r_i^2}{2}\right] = \\
 \frac{r_2}{4(\gamma - 1)} \left[ \frac{2 \gamma ln^2(\gamma)}{\gamma^2 -1} -\frac{\gamma^2 -1}{2 \gamma}\right]
\label{eq:K_alpha}
\end{multline}\\

\begin{equation}
K_{\beta} = \frac{r_i}{2(r_2 - r_i)} \left[ \frac{2r_2^2 }{r_2^2 - r_i^2}ln(\frac{r_2}{r_i}) - 1 \right] = \frac{1}{2(\gamma - 1)} \left[ \frac{2 \gamma^2 ln(\gamma)}{\gamma^2 -1} -1 \right]
\label{eq:K_beta}
\end{equation}\\

\noindent The force expressions for the viscous and pressure force will be compared with the results from the force calculations done during the simulations.
