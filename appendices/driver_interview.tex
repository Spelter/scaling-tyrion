\chapter{Interviews}
\clearpage

\section*{Driver interview}
\label{interview:bakkom}
Date: 2013-09-25\\
Subject: Ole Edvard Bakkom\\
Interviewers: Magnus Krane and Magnus Bae\\
\\
The purpose of of this interview was two-fold: (1) To gain knowledge about what types
of information that a Formula Student class racecar-driver consumes during typical events 
under an FS-competion. (2) To discuss what type of information that would be feasible for the driver
to consume in different circumstances; racing, training, renting a high performance vehicle for a limited time.


\begin{dialogue}
    \speak{Magnus} During a race how much information do you manage to consume from the instrument panel? 
    \speak{Ole} Not much, maybe a blinking LED or two, but it's difficult to read the speedometer for instance. 
    	The pace is quite high, and very few straights. This makes it difficult to be able to look down
    	and focus on complex information. 
    \speak{Magnus} Do you think you'd be able to consume more information in competitive context if the information was 
    	presented in a way that you didn't have to shift your focus away from where you're looking, for instance a heads-up-display?
    \speak{Ole} Yes, I think that would be a considerable improvement over last year's solution, as long as it's not obtrusive.
    	Although the usage in the competition might be a bit limited, I think it could be amazing for testing.
    \speak{Magnus} How do you think it would contribute in testing?
    \speak{Ole} Well, there's lots of things, but the first that comes to mind is, since we're building an electric car this year,
    	charging level - to get a notification when the battery needs to be charged or battery percentage.\\
    	Second the ability to have two-way communication with the rest of the crew. Last year we had to stop whenever they wanted to
    	give a message.\\
    	Lap-times would definitively make a difference during testing. Previously we've had to wait until we switched drivers to get a list of our lap times.\\
    	Cone-hits, in the competition we get added times for every cone we hit(flip?), getting real-time or close to it feedback if we hit a cone would be very practical.
    \speak{Magnus} What about in the competition? Which of these would be important? Two-way communication maybe?
    \speak{Ole} In the competition we're not allowed to have two-way communication with the vehicle or driver. The vehicle is allowed to send, but not receive.
    	With reservations that they might have changed the rules for 2014.
    \speak{Magnus} So no two-way communication. Can you think of anything that could be valuable for the competion?
    \speak{Ole} Well, we're adding brake pressure adjustment, and showing data or settings for it could be very practical. Using data from wheel sensors
    	maybe you could show an alert if a wheel locks up as well, hinting the driver to adjust brake balance
    \speak{Magnus} Interesting, I'm automatically thinking Gran Turismo 5-style info. Basically graphics showing the four tires, and changing color to red
    	when one locks up.
    \speak{Ole} That could work. There's also been talk about monitoring tire temperature using infrared sensors.
    \speak{Magnus} There's also the business case; rental for the weekend-racing. What type of information do you think could be interesting in that
    	context, something that would sell well with the judges? We've been thinking about creating an optimal route for the track using GPS, and then showing deviance.
    \speak{Ole} That sounds like a good idea. How about showing ghost data? Best or last lap would be really cool. I think that would really make testing better too. 
    	If you could have a laptimer showing as well that would be a good motivator. 
    \speak{Magnus} Maybe having real-time difference between previous (or best) lap could work as well? And how about sectoring it, comparing on a corner-per-corner basis
    	 
    \speak{Ole} Definitely. If I could compare myself lap by lap I would definitely push myself harder. Trying out different racing lines and corner speeds would give 
    	instant feedback. That's something that I think would really be cool, both for training and the business case.\\
    	Using GPS-data, could you show optimal speed through a corner? If you know the layout of the track that could probably calculated? Alternatively comparing speeds through
    	corners with your own best speed. 
    \speak{Magnus} Interesting. Calculating theoretical optimum speed might be far from reality? How about giving achievments if your laptime is 13 seconds and 37 hundreds?
    	
    \speak{Ole} Hehe, achievments could be cool. Even if the calculated max speed is off you'd get an indication. If you go through a corner at 50 and max speed 
    	could be 70 it should mean that you can push harder.
    \speak{Magnus} Good point. How about drifting, showing you how far you drifted around a corner?
    \speak{Ole} The cars don't really drift well, but maybe showing if you're close to optimum slip could be quite nice. Maybe a G-diagram, 
    	showing how much G's you're getting through corners. 
    \speak{Magnus} Noted. Could be cool, using data from accelerometers maybe. So if you were to prioritize, what would you put as the number one
    	priority? 
    \speak{Ole} That would definitively be for testing, lap times and indication of difference is what I think would add the most value during testing and training.
    \speak{Magnus} Ok. What do you think would be the biggest benefits of such a system?
    \speak{Ole} I think one benefit would be that it would make testing more effective. Second, it could make it much easier to push your own performance. 
    \speak{Magnus} Thank you for the interview, I feel we've gotten a lot of insight and good ideas here. If you think of anything else please let us know. 
    \speak{Ole} I will see if I can come up with something. 
    
  \end{dialogue} 


  The interview transcript was approved by Ole Edvard Bakkom on 2013-10-10.
