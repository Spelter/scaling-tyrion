\addcontentsline{toc}{section}{Abstract}
\section*{Abstract}

In a complex system such as the Norwegian railway network, there are much that
can affect a trains punctuality. The undertakers strive to achieve higher and
higher punctuality, while the infrastructure owner, Jernbaneverket, strive for
minimal downtime on the railway network. There is collected much data for 
analysis about the trains run and the infrastructure, in order to achieve 
higher punctuality and less downtime. The users are able to track down the 
source of delays and find possible improvements on the infrastructure, by 
analyzing and comparing the different data sets collected.

There are many users across both different companies and internal divisions in
a company that need to cooperate, due to the size and complexity of a railway
network. The different users have different needs when studying the data sets.
A area director have the need to see the big picture over time, while a segment
director wants to see every detail within its segment. \\

In this thesis we demonstrate a system that is aware of the different
stakeholders requirements when presenting data. The system also takes into
consideration the stakeholders need for analyzing different types of data, and
comparing these. 

Finally, we conclude how users should be defined within a domain in order to 
be aggregated over.

\clearpage
