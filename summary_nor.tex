\addcontentsline{toc}{section}{Norwegian Abstract}
\section*{Sammendrag (Norwegian Abstract)}

I et komplekst system som det norske jernbane nettverket er det mye som kan
påvirke punkligheten til et tog. Operatørene på jernbanen streber etter å oppnå
høyere og høyere punklighet, mens infrastruktur eieren, Jernbaneverket,  
streber etter minst mulig nedetid på trafikk nettverket. For å kunne oppnå 
høyere punklighet og mindre nedetid, blir det samlet inn mye data om kjørte 
tog og infrastrukturen for å kunne analysere. Ved å analysere og sammenligne forskjellige de innsamlede data settene er det mulig å forstå grunnen til forsinkelsene.

Grunnet størrelsen og kompleksiteten på et jernbane nettverk er det mange 
brukere på tvers av selskaper og interne avdelinger som må samarbeide for å gi 
et best mulig togtilbud. De ulike brukerne har forskjellige behov når de 
studerer data settene. En områdedirektør har behovet for å kunne se de store 
trekkene over lengre tid innenfor området, mens en strekningsdirektør vil se 
alle hendelser på sin strekning.\\

I denne masteroppgaven demonstrerer vi et system som tar hensyn til de
forskjellige brukerene sine behov for forskjellig presentasjon av data.
Systemet tar også hensyn til behovene for å kunne se på forskjellige data
og sammenligne disse.

Til slutt konkluderer vi med hvordan brukerne bør bli definert i forhold til
domenet for å kunne aggregeres over.
	
\clearpage
