
\RequirePackage[l2tabu, orthodox]{nag}
\usepackage[english]{babel}
\usepackage[utf8]{inputenc}
\usepackage[T1]{fontenc}
\usepackage{float}
\usepackage[font=small,labelfont=bf]{caption}
\usepackage{fancyhdr}
\usepackage{times}
\usepackage{subcaption}
\usepackage{multirow}
\usepackage{tabularx}
\usepackage{perpage} % For å få fotnotenummerering til å starte på nytt for hver side
\usepackage{graphicx}
\graphicspath{{grafikk/}}
\usepackage[export]{adjustbox} %In including a figure, add "center" as a flag

\usepackage{slashbox}
\usepackage{epstopdf}
\usepackage{varioref}
\usepackage{listings} % For å vise frem kode på en bra måte
\usepackage{textgreek}
%\usepackage{tikz,pgfplots} % For å plotte Matlab figurer på en god måte
\usepackage{setspace}
\usepackage{amssymb}
\usepackage{mathrsfs}
\usepackage{amsthm}
\usepackage{amsmath} % For å tilpasse ligninger til å vises over flere linjer
\usepackage{pdfpages} % For å legge til pdf i appendix delen
\usepackage{rotating}
\usepackage[Lenny]{fncychap}
\usepackage[pdftex,bookmarks=true]{hyperref}
\usepackage[pdftex]{hyperref}
\hypersetup{
    colorlinks,%
    citecolor=black,%
    filecolor=black,%
    linkcolor=black,%
    urlcolor=black
}

\usepackage{mathtools}
\usepackage{color} % Farger i dokumentet

\usepackage{dialogue}
\usepackage[section]{placeins}
\usepackage{microtype}

\linespread{1.5}

% \lstset{frame=shadowbox, rulesepcolor=\color{black}}
\lstset{numbers=left, frame=single, tabsize=2, breaklines=true}

\DeclarePairedDelimiter\abs{\lvert}{\rvert}%
\DeclarePairedDelimiter\norm{\lVert}{\rVert}%

\makeatletter
\let\oldabs\abs
\def\abs{\@ifstar{\oldabs}{\oldabs*}}
%
\let\oldnorm\norm
\def\norm{\@ifstar{\oldnorm}{\oldnorm*}}
\makeatother


\DeclareMathSymbol{\comma}{\mathpunct}{letters}{"3B} % Legger inn kommandoen \comma som komma i math mode

\hypersetup{%
    pdfborder = {0 0 0}
}

%\usepackage{endnotes}
%\usepackage[portrait, pdftex]{geometry}

\newcommand{\HRule}{\rule{\linewidth}{0.5mm}}
\pretolerance = 1414
\tolerance = 1414

\MakePerPage{footnote} % For å få fotnotenummerering til å starte på nytt for hver side
\labelformat{equation}{equation~(#1)}
\labelformat{figure}{figure~{#1}}
\labelformat{subfigure}{figure~\thefigure #1}
\labelformat{table}{table~{#1}}
\labelformat{section}{section~{#1}}
\labelformat{subsection}{section~{#1}}
\labelformat{chapter}{chapter~{#1}}


\pagestyle{fancy}
\fancyhf{}
\renewcommand{\chaptermark}[1]{\markboth{\chaptername\ \thechapter.\ #1}{}}
\renewcommand{\sectionmark}[1]{\markright{\thesection\ #1}}
\renewcommand{\headrulewidth}{0.1ex}
\renewcommand{\footrulewidth}{0.1ex}
\fancypagestyle{plain}{\fancyhf{}\fancyfoot[LE,RO]{\thepage}\renewcommand{\headrulewidth}{0ex}}
