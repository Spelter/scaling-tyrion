% !TEX root=../thesis.tex
\chapter{Method}
\label{chapter:method}

In working with this project the group has undertaken a considerable study of
literature around both software architecture, software metrics and assessment, electronics, display technology,
and studies about driver environments and performance. The literature has
served as a platform to support our project goals of creating an enabling
software architecture. The group has also conducted an interview with a driver
from last years Revolve-team. In collaboration with Revolve we have elicited
requirements and use-case scenarios which in turn has resulted in our proposed
architecture design (including design artifacts) and assessment guidelines.

The biggest weakness of the architecture described herein is the lack of
verification. First, during the project period we didn't have access to any hardware
similar to the one the system will be running on. Second, there are still some
uncertainties with regards to the amount of data that is to be handled by the
system due to the fact that sensor update rates and CAN-bus package format is
still an unknown factor. We have however developed a proof-of-concept desktop
application to simulate a configuration environment for the proposed system.
We will be discussing this application further in \vref{chapter:results}. 

\section{Literature Study}
To acquire a broad picture of the problem area and domain the group has studied
a varied set of literature ranging from research articles to books on various topics. Some of the most useful literature we have read includes the following:
\begin{itemize}
	\item ``An Introduction to Object-Oriented Analysis and Design and Iterative Development'' by Craig Larman \cite{Larman:UML}
	\item ``Software Architecture in Practice'' by Bass, L. and Clements, P. and Kazman, R.
	 \cite{bass2012software}
	\item ``Designing embedded hardware [Kindle Edition]'' by John Catsoulis
	\item ``A software architecture evaluation model'' by Dueñas, Juan C and de Oliveira, William L and Juan, A.
	\cite{duenas1998software}
\end{itemize} 
For an exhaustive list of all literature used we refer to the
bibliography\vpageref{bibliography}.

\section{Architecture Design Method}
In the process of designing this architecture we have opted to use design
artifacts and methods described by Craig Larman \cite{Larman:UML}. We started
out at the beginning with performing an interview with one of last years 
drivers (\vref{interview:bakkom}) to get a picture of how the cockpit 
experience is for the driver during
an event. We also got some interesting ideas to build on. Building on this
information we arranged an informal requirements workshop together with a few 
members of this years Revolve-team; eliciting both quality requirements and
functional requirements (and lots of ambitious ideas). One important benefit of
bringing experienced members in on the workshop was that we could easily
discover external requirements originating in the FS competition rules \cite{fsae:2014rules}.
Both Larman and Bass et al. suggests that a systems
architecture should be based mostly on non-functional and quality requirements,
especially those that strongly influence the architecture.
\cite[p.20]{bass2012software}\cite[pp. 541-554]{Larman:UML}

After the workshop we started working on refining and defining the results,
writing up use-cases in a terse format. Larman suggests \cite[pp. 95-95]{Larman:UML}
that during early phases of development most use cases (except the ones that 
might be architecturally significant or with high risk) should be kept terse 
and rather be expanded upon and refined when development is underway. Since no
architectural development was performed during this project we decided to spend
more time trying to clarify them and use the results to help with identifying
the architectural requirements.

In the further process we listed the discovered requirements in an
architectural factors table \cite[pp. 56-59, 545-547]{Larman:UML}. The goal of
this activity was to define measures that can be used to measure the
architecture's ability to meet the goals and requirements, and to give an
easily accessible ranking of importance for these requirements.

The last artifact to be developed was the system architecture document \cite[p. 550]{Larman:UML} where we
recorded and detailed the suggested solutions, and their reasoning, to resolve the architectural
factors.

In working with the architecture design we have tried to work in an iterative
fashion, but due to the compact nature of the project some processes bear the
look and feel of a ``waterfall'' process. We do not intend that the
architectural design suggestion presented in this project to be the final one, 
but rather its intended to be a good
starting point for development and refinement. We also suggest metrics that can
be used to measure if the architecture meets the goals set, and believe it is important
that the control process starts early in the development to see if the
architecture meets the requirements or needs further work 
\cite[pp. 556-557]{Larman:UML}.

In describing our architecture we have opted to use UML, or modeling close to
UML where we felt that it increased readability, or where we were pressed for time and
UML compliance wasn't deemed the most important, planning to revise this when
further development is being undertaken. We chose to not use any ADLs to describe the
system, in part because that would be worthy a project of its own, and in part because we 
are not interested in traces of the system 
\cite[p. 42-43]{Roscoe:2005:TPC:550448},
but in the systems ability to meet its critical requirements. 



