% !TEX root=../thesis.tex
\chapter{Discussion}
\label{chapter:discussion}

In this section we will discuss the results presented in the previous chapter.
We will explain why we have made the choices we made, and point to relevant
literature. 

\section{Stakeholders}
As mentioned in section \vref{results:stakeholders} we decided to include
fictive stakeholders to get a wider picture of for whom such a system would
have benefit and what goals they might have. The listings in the appendices are
not meant to be exhaustive, and we have no guarantee that the intentions and
goals we have attributed to these fictive stakeholders are realistic. On the
other hand we would like to argue that having this expanded view of
stakeholders have made our system requirements more generic and we believe that
the system, once developed, could easily be used outside of a motorsport
context. 

\section{Hardware}
Software and hardware go hand in hand, but many software engineers are used to
working with so many layers of abstraction between the run-time environment 
and the hardware that they don't have to think about it. Our proposed system will require low-level programming and
interaction, as well as hardware design and integration.

In section \vref{results:wearableHud} we suggested that the best way to create
a better driver information system was to utilize head-up technology. This is
based on the various studies we've read; many experiments have shown that HUD 
technology gives faster driver response - and more consistent speed control. 
\cite{Liu:HudPerformance,lim1999heads,cheng2007intelligentVehicles}. This, in 
combination with our own findings (appendix \vref{interview:bakkom}) makes us 
convinced that a head-up system is by far the best approach.

In section we proposed a wearable display device to
create a HUD-experience for our driver information system. Another alternative
was to go for a helmet mounted device, but that requires a specially
manufactured and approved helmet. When we reached out to the Formula Student
United Kingdom organizers for a clarification on the helmet rules we got the
following answer:
\begin{quote}
After discussion with the MSA who would have to approve this as they check all of the helmets, mounting anything to the helmet is prohibited unless the helmet is homologated to have this mounted by the manufacturer, so that rules out one of the options.

For the spectacle option, there is no regulations regarding the wearing of spectacles  other than that they shouldn’t interfere with any safety equipment, shouldn’t interfere with the colour of signal flags and of course should be made from material that is safe, so if the second device , http://www.google.com/glass/start/what-it-does/ , can be deemed to fit that criteria then this should be ok.
 
Also, If it is connected to the vehicle by some form of cable, then of course, this must not interfere with the safe evacuation of the vehicle in an emergency.
I hope this helps.
\end{quote}

This reduced our alternatives to have a fixed HUD in front of the steering
wheel or using a spectacle-based solution which will provide the driver with a
display that is fixed relative to his head and where the display elements
reside in the drivers field of view or peripheral vision (depending on
configuration and device possibilities). In 2007 Cheng et al. \cite{cheng2007intelligentVehicles} found that movement in the peripheral vision was enough to
attract car drivers' attention, this can be used for purposes such as warnings or messages. 

Choosing to make the system cable free was a natural result from FSAE rules
article 4.8:
	\begin{quote}
		All drivers must be able to exit to the side of the vehicle in no more than 5 seconds. Egress time 
		begins with the driver in the fully seated position, hands in driving position on the connected steering 
		wheel and wearing the required driver equipment. Egress time will stop when the driver has both feet 
		on the pavement. \cite{fsae:2014rules}
	\end{quote}

\section{Requirements and Software Architecture}

The most difficult task in developing this system is most likely going to be able to
meet the requirements of being a real-time system while fulfilling other
high-level goals and requirements. The system will have to mix efficient code
with levels of abstraction and modularization that meets the system
requirements and still have an acceptable latency. One important aspect of
achieving high modifiability is to have high cohesion and low coupling between
modules \cite[p. 123]{bass2012software}. 

Bass et al. \cite[p. 118]{bass2012software} suggests that creating a UI-
builder tool can be used as a tactic to reduce cost when trying to achieve 
high modifiability. Our experience with such tools are also positive and that
is why we suggest implementing such a tool to be able to easily generate and 
verify layout configurations, visualize layouts before deploying, and ideally 
handle more of the configuration as well. 

Even though we have scoured literature about architecture design
\cite{goebl2007realtimecapable,bass2012software,duenas1998software,magee1995specifying,Allen97Thesis,Larman:UML}
(and many more that we haven't even written down) we have found that the
textbooks (Larman and Bass et al.) are the ones that have provided the best
grounds for building our own architecture. Part of this might be because our
proposed system differs from the architectures described in other papers, but
it also seems that many systems do employ variations of the patterns and
tactics described by Larman and Bass et al. 

In our design proposal we have tried to address the key architectural factors,
in a way that is simple and yet complex enough to support our goals. The
design has not been test or verified, so we cannot say whether or not its a
success, and we expect that further refinement will have to be done if the
system gets developed. We propose our architecture design as a starting point
to build from and upon. 

\section{Proof-of-concept configuration tool for display layout}
As mentioned in section \vref{results:poc} we created a prototype of a
configuration tool for display layout. The prototype worked well and enabled us
to mock up configurations in short amounts of time. However we do think that
creating a stand-alone application with Qt might be overkill if we can build
upon existing tools like Eclipse \cite{eclipse} to rapidly develop a tool to
create configuration files, but one factor to consider is previous knowledge
about tools and getting started with building a functional eclipse-plugin might
take some time, while we already have the shell for a Qt-powered configuration
application. 

\section{Verification of design}
The proposed design has not gone through any verification steps, but we suggest
using the proposed metrics as a starting point to measure the systems ability
to meet requirements and goals. 
