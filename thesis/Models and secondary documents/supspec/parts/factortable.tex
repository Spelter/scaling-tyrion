% !TEX root=../document.tex

\section{Architectural Factors}

This section contains descriptions of the identified architecturally significant requirements and
suggested quality measures. We also give the current and expected variability
on these factors and prioritize and give an assumed risk/difficulty factor.

\begin{table}
\small
\begin{tabularx}{\textwidth}{|X|X|X|X|l|l|}
	\hline
	\rowcolor{gray!50}
	Factor & Measures and quality scenarios 
		& Variability (current flexibility and future evolution)
			& Impact of factor (and its variability) on stakeholders, 
			architecture and other factors
				& \parbox[t]{1cm}{Prior-\newline
				ity for\newline
				Suc-\newline
				cess}
					& \parbox[t]{1cm}{Diffi-\newline
					culty\newline
					or\newline
					Risk} \\
	\hline
	\multicolumn{6}{|l|}{\textbf{Maintainability -- Modifiability}} \\
	\hline
	Rapid reconfiguration of display layout 
		& A reconfiguration of display layout should take less than 10 minutes,
		including system reconfiguration.
			& Current flexibility - We do not know which layouts will be advantageous and being able to rapidly move things around and display concept to drivers should prove positive. \newline
			\newline
			evolution - Requirements may change over time and usage scenario.
				& High impact on the large scale design.\newline
				\newline
				Important to be able to rapidly test various layouts during
				development.\newline
				\newline
				Important for potential owners of system to have an easy way of configuring
				the system.
					& H & M \\
	\hline
	Configuration of sensor calibration without compilation of code.
		& Calibrating an average sensor should take less than 10 minutes.
			& Current flexibility - Calibration of sensors might change during project. \newline
			\newline
			evolution - Next years vehicle will most likely be utilizing other sensors.
				& Positive impact on modularity.\newline
				\newline
				Selling point.
					& M & H \\
	\hline
\end{tabularx}
\end{table}
\clearpage

\begin{table}
\small
\begin{tabularx}{\textwidth}{|X|X|X|X|l|l|}
	\hline
	\rowcolor{gray!50}
	Factor & Measures and quality scenarios 
		& Variability (current flexibility and future evolution)
			& Impact of factor (and its variability) on stakeholders, 
			architecture and other factors
				& \parbox[t]{1cm}{Prior-\newline
				ity for\newline
				Suc-\newline
				cess}
					& \parbox[t]{1cm}{Diffi-\newline
					culty\newline
					or\newline
					Risk} \\
	\hline
	\multicolumn{6}{|l|}{\textbf{Maintainability -- Testability and Modularity}} \\
	\hline
	Automated unit testing of software components
		& Software components should have a way of testing their
		interface functionalities through automated unit testing that reports errors.
			& Current flexibility - Some software components are difficult to test outside of their runtime environment due to different architectures. \newline
			\newline
			evolution - none
				& Positive impact on modularity.\newline
				\newline
				Provides implicit documentation of features and makes further development
				easier.
					& M & L \\
	\hline
	\multicolumn{6}{|l|}{\textbf{Portability -- Adaptability}} \\
	\hline
	Only low-level components should require re-programming or replacing for
	use with different network protocols or other hardware environments.
		& No reprogramming of high-level systems should be required for changing
		hardware or connectivity. 
			& Current flexibility - Not an issue during prototype development and
			testing\newline
			\newline
			evolution - Other vehicles might have different low-level hardware and
			software requirements.
				& Positive impact on modularity.\newline
				\newline
				Nice selling point for potential system integrators to be able to swap out technology without having to
				replace bulks of software componentes, and to be able to support different
				sensor networks or voltage requirements.
					& L & M \\
	\hline

\end{tabularx}
\end{table}
\clearpage

\begin{table}
\small
\begin{tabularx}{\textwidth}{|X|X|X|X|l|l|}
	\hline
	\rowcolor{gray!50}
	Factor & Measures and quality scenarios 
		& Variability (current flexibility and future evolution)
			& Impact of factor (and its variability) on stakeholders, 
			architecture and other factors
				& \parbox[t]{1cm}{Prior-\newline
				ity for\newline
				Suc-\newline
				cess}
					& \parbox[t]{1cm}{Diffi-\newline
					culty\newline
					or\newline
					Risk} \\
	\hline
	\multicolumn{6}{|l|}{\textbf{Maintainability -- Reusability}} \\
	\hline
	Rapid reconfiguration of sensor profiles
		& Creating a sensor configuration should take less than 2 hours for a
		complete setup.
			& Current flexibility - Sensors on the test-vehicle won't change much during project, and for the project this only needs to be configured once, then updated in case of any changes. \newline
			\newline
			evolution - If the system is to be installed to other vehicles this should
			be a process that doesn't require reprogramming of high-level components.
				& Impact on the large scale design.\newline
				\newline
				Provide value for next years Revolve team and provide a good platform for
				further development.\newline
				\newline
				Nice selling point for potential system integrators to have an easy way of configuring
				the system and assuring them that swapping out technology doesn't disable
				the system.
					& M & H \\
	\hline
	\multicolumn{6}{|l|}{\textbf{Performance -- Time behaviour}} \\
	\hline
	Processing and displaying data with minimal latency
		& The system shall be able to display acquired data with less than 200ms
		delays in at least 90\% of the time.
			& Current flexibility - none\newline
			\newline
			evolution - future hardware might allow longer pipelines due to increased
			computing performance.
				& High impact on the system design.\newline
				\newline
				System needs to meet performance requirements. Stale data is not
				interesting.
					& H & H \\
	\hline


\end{tabularx}
\end{table}
