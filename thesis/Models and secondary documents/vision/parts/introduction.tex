% !TEX root=../document.tex

\section{Introduction}

We envision a next generation driver information system, ReVision, that will utilize
state-of-the-art technology to deliver critical information to race-car drivers
in real-time during training and race activities. ReVision has the ability 
to be quickly adapted to new instrumentation and other vehicles, changing 
needs, requirements, and technology.

In this document we will present the stakeholders and their high-level goals for this system and
how this system should realize that vision.

\subsection{Definitions and abbreviations}
See project dictionary.

\section{Background}
To understand all the concepts discussed in this document and other design
documents it is necessary to have some knowledge about the domain of automotive
vehicles and automotive electronics. This section contains a quick introduction
to these areas.

\subsection{Automobile electronics}
Todays automobiles are complex machines when compared to 50 years ago when they
had an engine, transmission, a fuel system and a cooling system, some brakes
and a clutch. Modern cars are often equipped with systems for emissions control, adaptive suspension, ABS, traction control, airbags, and
intelligent gearboxes.

Both racing  and  personal vehicles are fitted with complex mechanical and electrical
systems being controlled mostly by small embedded computers. For these systems
to be able to know what they are doing they need input data about the state of
the surrounding and controlled systems. This data is gathered by means of
sensors registering real-world factors like position, temperature, pressure,
and flow.

These systems consists of many components located at various places within the
car. Sensors that detect collisions and wheel speed are examples of sensors
often fitted very far away from their control units. Some sensors are often
much closer, but it is still difficult to contain a system to a small area of a
vehicle.

Wires are basically antennas and are susceptible to catching induced currents
from magnetic fields, and also emits their own magnetic fields when current is
applied. Petrol engines use high-voltage components that may create a lot of
noise if not shielded properly. This often results in noisy electrical
environments, something even modern cars struggle with. Automobile electronics
have to be designed to tolerate very high transient voltages. 

\subsection{Sensors and control networks}
Traditionally sensors used to be directly wired to the control unit, but as
system complexity and the
number of sensors grew this became increasingly difficult to handle. The
solution was to utilize digital communication buses which allowed components to talk with each other and reduced the amounts of wires considerably. The most significant bus-standard is the Controller Area Network bus
(``CAN-bus'')
that was developed specifically to be used for automotive purposes and is
very error-resilient \cite{can-appnote}.

In a CAN system connected nodes communicate by sending messages ('frames').
Frames have an id which is also the priority of the message (lower number =
higher priority), if two nodes attempt to send at the same time the frame
with the highest priority will be transported over the bus, the losing node
will retry when the bus is clear. \cite{can-appnote}

Most sensors are in their very nature analog and thus requires their output
values to be represented digitally. Although some modern sensors come with
digital outputs, most sensors requires peripheral components that adapt an
output voltage to a level suitable for digitalizing before an analog-to-digital
converter reads the voltage and can communicate it to other digital components,
most commonly a micro controller unit (MCU). The MCU would then take the value
from the sensor, wrap it in a CAN-frame and transmit it on the
CAN-bus, either at its own accord or when requested from another node. \cite[loc. 5020-5203,
5732-7565]{catsoulis:embedded}\cite{can-appnote}

\section{Positioning}

\subsection{Business Opportunity}
Existing driver information systems (DIS) are usually implemented as dashboards
that requires the driver to look down. Even in motorsport this still rings true
today. Racing is an activity with a high pace and there is little time to look
at instruments and dials. Traditionally these systems are therefore simple in nature, and
tailor-built for the car they are mounted in. They indicate activity by using
lamps or big numerical displays or gauges. These design are static and requires
the driver to remove focus from the road. Using state-of-the-art technology
gives the possibility to create a flexible solution which resides in the
drivers field-of-view, a so-called Heads-up Display (HUD), that overcomes the
challenges plaguing traditional solutions. 

\subsection{Problem Statement}
Traditional dashboard solutions are static and inflexible. In compact one-seat race cars they take up significant amounts of space that could be
repurposed, or removed to lower weight. Moving one system from one car to another is not even part of the
design considerations. Traditional systems are designed with that car in mind.
During competition driving these systems are almost never used; the driver
isn't able to perceive much information during racing, and there is no way to
give or visualize complex information \cite[Appendix A]{kranebae:fordypning}
. They don't accommodate the varying needs
of various user groups. A car that is being used in a training context could 
benefit from different configurations than one being used in a competition 
(race) context. 

\subsection{Product Position Statement}
ReVision is a computerized information system that is easy to adapt to
different demands and requirements, portable and flexible. It can be tailored
to the varying needs of different target groups, from novice amateur
racers to professionals trying to get the most out of the car during a serious
competition.

\subsection{Alternatives and Competition}
Today there exists very few solutions like this on the open market \cite{motionResearch:sportvue}, although the time seems to be ripe; 2013 has seen a 
flood of prototypes \cite{skully,caranddriver:nissan_3E,cycleworld:nuviz,popularscience:livemap}. They are, however, all going after the consumer 
market, and most are targeted towards motorcyclists.
With ReVision we are aiming to cater for professional and amateur racers. 



