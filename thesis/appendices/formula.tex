% !TEX root=../thesis.tex

\chapter{The Formula Student Competitions}
\label{appendix:formula}
\clearpage
The Formula Student competitions is a series of international competitions 
between engineering students where teams compete with small formula-style cars 
they have designed and built from scratch:

\begin{quote}
Your team is tasked to produce a prototype for a single-seat race car for auto-cross or sprint racing, and 
present it to a hypothetical manufacturing firm. The car must be low in cost, easy to maintain, and reliable, with high 
performance in terms of its acceleration, braking, and  handling qualities. During the competition your team must demonstrate 
the logic behind your proposal and must be able to demonstrate that it can  support a viable business model for both parties.
 \cite{FS:challenge}
%(http://www.formulastudent.com/formula-student/about-us/thechallenge)
\end{quote}

\noindent Any FS competition is built up with 3 static and 5 dynamic events. The static events is where the teams must defend their design 
and solutions on the car; why the car has cost as much or little as it did, and try to sell it to investors through a 
business plan. The dynamic events is meant to test the abilities of the car in both cornering and acceleration, 
durability, and fuel-efficiency.

\section{Static Events}
Engineering Design Event: 
An eight page Design Report is being reviewed by the judges followed by an inspection of the constructions and discussion 
with the students. This is where the teams defend every part of the car. Why designs, parts, etc. were chosen. 
A maximum of 150 points can be awarded. \cite{FSG:disciplines}

Cost and Manufacturing Event:
Based on the Cost Report, which details the cost of the prototype, the teams discuss every aspect of the car with the judges 
why the cost estimate ended up as it did. The cost report is based on a list of all components, where every item has a cost 
based on what the part is made of, how it is made, and how it is put together. These are not the real costs of the part, but
synthetic and standardized values close to the cost that would be incurred during manufacturing. A maximum of 100 points can be awarded. \cite{FSG:disciplines}

Business Presentation Event:
 The team gives a 10 minute presentation where the judges pretend to be representatives of a fictional manufacturer  and the 
 point is to show that the car is suited for the target group (nonproffesional weekend autocross racing). During this 
 presentation the focus is on value and sell-ability. A maximum of 75 points can be awarded. \cite{FSG:disciplines}
  %cite{http://www.formulastudent.de/fsg/about/disciplines/}

\section{Dynamic Events}
Acceleration: This is where the car's acceleration gets tested. The event is based on a straight line race on 75 meters of track. A maximum 
of 75 points can be awarded. \cite{FSG:disciplines}

Skid-pad: This event tests how much lateral acceleration the cars can produce. The course is built in the shape of a figure of eight. 
The car goes round each circle twice and the last of the rounds on each circle is timed, the total time is the average of those 
two times. A maximum of 100 points can be awarded. \cite{FSG:disciplines}

Auto-cross: The auto-cross event tests the cars driving dynamics and handling qualities. 
The cars race on a one kilometer course composed of straights and curves.
A maximum of 100 points can be awarded. \cite{FSG:disciplines}

Endurance: This event tests the durability of the car over a 22 kilometer long course. Every aspect of the cars physical capabilities 
gets put to the test during this event, such as acceleration, handling and reliability. A maximum of 325 points can be awarded. \cite{FSG:disciplines}

Fuel efficiency: The fuel efficiency of the car is calculated based on average fuel consumption per completed lap. This event is calculated 
based on the endurance event, and the teams must complete the driver exchange before the fuel efficiency is allowed to be 
calculated. A maximum of 100 points can be awarded. \cite{FSG:disciplines}\\
