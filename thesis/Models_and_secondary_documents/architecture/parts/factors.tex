% !TEX root=../document.tex

\section{Architectural Factors}

This sections purports to record the architectural decisions made and the
reasoning behind.

\subsection*{Technical Memo}
\subsection{Issue: Maintainability -- Rapid reconfiguration of display layout}
\textbf{Solution Summary: Generated layout-configuration file in JSON format
from GUI-tool.\\
\\
Factors:}
\begin{itemize}
\item Generated configuration-file avoids human errors in configuration.
\item Configuration is verified by GUI-tool.
\item Reduced time spent on configuration
\item Human readable format with little overhead.
\item Pre-made widgets allow for easy placement and configuration of variables.
\end{itemize}
\textbf{Solution}\\
\\
Achieve reduced cost in effort and time used on configuration by developing a
GUI tool that generates a configuration file that can be uploaded to the DIS.
Using a widget-solution allows for experimentation with different presentations
of various sensors, as well as making it easier to find configurations that fit
together.\\
\\
\textbf{Motivation}\\
\\
We don't know which design and layouts will give positive results, and
developing this tool and specifying a format contract beforehand reduces time 
spent debugging and makes it easier to quickly test concepts beforehand.\\
\\
\textbf{Unresolved Issues}\\
\\
- Contract for format has to specified.\\
\\
\textbf{Alternatives Considered}\\
\\
- Manual configuration of files. Takes a lot of time to write by hand, needs 
testing to see if they work.\\
- Using XML instead of JSON. Has larger overhead and more difficult to parse.\\
- Hard coding configuration in software. Gives little flexibility,
might increase performance. 



\subsection*{Technical Memo}
\subsection{Issue: Maintainability -- Configuration of sensor calibration without compilation of code}
\textbf{Solution Summary: Using GUI-tool and reconfigurable pipe and filter
pattern to make configurable transformation filters\\
\\
Factors:}
\begin{itemize}
\item Generated configuration-file allows verification before runtime.
\item Generalized pipe and filter pattern allows pipeline transformations of
data.
\item Filter-pipeline allows for easy debugging of transformations.
\end{itemize}
\textbf{Solution}\\
\\
Achieve reduced cost in effort and time used on configuration by developing a
GUI tool that generates a configuration file that can be uploaded to DIS, it is
also possible to do this in configuration files, but format might be complex
due to mathematical notation. GUI tool could also provide visualization of
transformations.\\
\\
\textbf{Motivation}\\
\\
Sensor data needs to transformed to meaningful real-world data so that it can
be visualized. This fits very well with the pipe and filter pattern and data
can be transformed with minimal latency.\\
\\
\textbf{Unresolved Issues}\\
\\
- Contract for configuration format has to be specified.\\
- Is the benefit of making a GUI-tool great enough to be worth the extra
effort?\\
- Will the overhead from running potentially long and concurrent pipelines
result in unacceptable latency (ie. do we have enough computing resources to
allow for this type of reconfigurability)?\\
\\
\textbf{Alternatives Considered}\\
\\
- Manual configuration of files. See ``Solution''\\
- Hard coding configuration in software. Gives little flexibility,
might increase throughput and decrease latency. \\


\subsection*{Technical Memo}
\subsection{Issue: Maintainability -- Rapid reconfiguration of sensor profiles
}
\textbf{Solution Summary: Using abstraction and a layered architecture to defer
network package specification to higher levels of software and allow for
configuration files specifying package format.\\
\\
Factors:}
\begin{itemize}
\item Layer pattern allows compartmentalizing functionality and deferred
configuration. 
\item Increases modularity and reusability.
\item Increases overhead related to network computing.
\item Increases system complexity
\item Allows for easily using multiple package formats within the same
setup.
\end{itemize}
\textbf{Solution}\\
\\
Develop a format specification for network packets decomposition and
composition to allow high level configuration of low-level network systems.
Using this approach makes it possible to easily adapt the system to other
vehicles or if the package format should change. It also makes it easier to
support different network standards as well.\\
\\
\textbf{Motivation}\\
\\
Making it easier to port the system to future vehicles and use other network
standards.\\
\\
\textbf{Unresolved Issues}\\
\\
- Format for configuration file.\\
\\
\textbf{Alternatives Considered}\\
\\
- Hard coding configuration in software. Gives little flexibility,
might increase throughput and decrease latency. \\



\subsection*{Technical Memo}
\subsection{Issue: Performance -- Processing and displaying data with minimal latency
}
\textbf{Solution Summary: Lowering the level of abstraction in time-critical
code and reducing the number of layers where layer pattern is applied. Keeping
length of pipelines to a minimum where pipe and filter pattern is applied.\\
\\
Factors:}
\begin{itemize}
\item Layer pattern increases overhead and lowers performance.
\item Long pipelines increases overhead.
\item High number of pipelines increases overhead.
\item Tailored filters can reduce overhead by replacing multiple filters with a
single one.
\item Direct access to lower layers for certain functionality can increase
performance where functionality is being used directly.
\item Optimizing mathematical functions can greatly increase speed as both
multiplication and division typically requires considerably more compute time 
than addition and subtraction.
\end{itemize}
\textbf{Solution}\\
\\
Start early with performance testing and use automated performance testing code
commits. Performance also needs to be tested on actual hardware as soon as
possible. Reducing the level of abstraction will also result in increased
performance, and complex calculations need to be optimized for the computation
environment.\\
\\
\textbf{Motivation}\\
\\
Keep the system useful by providing data in real-time.\\
\\
\textbf{Unresolved Issues}\\
\\
none\\
\\
\textbf{Alternatives Considered}\\
\\
- Performance bottlenecks can be overcome by using more powerful hardware.
- Splitting calculations between computing nodes can also result in lower
latency. 
