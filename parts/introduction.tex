% !TEX root=../thesis.tex

\chapter{Introduction}

In this report we aim to design a software architecture for a driver information
system\footnote{Also known as In Vehicle Information System, IVIS}
that is to be prototyped and installed in a formula student race car. We will
also discuss the state of the art, technological alternatives and introduce
background information pertaining to the domain and use-cases.

To design this architecture we have undertaken a significant literature study
of both architectural design patterns and driver/operator information systems.
We have looked at what a typical driver information system does, how it does
it, and what the state of the art is. Other fields employing information 
systems to provide vehicle operators with real-time information have also been
studied.

This research effort is made to serve as a foundation to build a prototype
driver information system in collaboration with Revolve NTNU (Revolve), a 
student organization  created by students at the Norwegian University of 
Science and Technology.
The the collaboration with Revolve will serve as a testing ground for our 
design and allow us to test its feasibility in the
context of motorsport.

For the last three years Revolve has been hard at work building formula style race
cars to participate in events of the international motor sport competition series
``Formula Student'' (FS). The competitions are arranged by the Institution of Mechanical Engineers. In the competitions teams of students compete against each other
with small formula-style cars they have designed and built from scratch. There are
disciplines that involves racing the cars as well as disciplines in business,
design, and cost. The combined score from these events decide the official
results, but there are also prices awarded for excellence in different fields

Motorsport has for many years been a testbed for car manufacturers and a lot of
technology that we today take for granted has come to fruition through motorsport. However
ordinary cars have seen larger leaps in technology with regards to information
consumption than cars built for motor racing. For some years some high-end cars
have even been equipped with heads-up displays (HUD) that allows the driver to 
keep focus on the road instead of looking down at the instruments \cite{wiki:hud},
and it is becoming more and more common.

Despite great developments in automobile technology over the past 50 years little
has changed in the way that motorsport drivers receives information during an
event. There are displays, gauges, and other visual indicators\footnotemark[1]
, in addition to radio-communication \cite{wiki:formula_radio} if that is allowed in the competition. The technology under the hood
has become much more advanced, but the amount of information you can
convey to the driver still seems to be very limited; the pace is just too high
to be able to keep moving focus from the road (see appendix \vref{interview:bakkom}).

Driver information systems in the context of motorsport is a field neglected by
science (see \vref{background:rel_reasearch}). There has been done some
research around driver information systems in commercial vehicles and normal 
passenger cars, most of which involve studies of heads-up display technology or
use.

This is therefore a field where we are threading into new territory. 
Requirements are hard to elicit up front, and we don't know
from the start what is a good design for such a system. It is quite possible
that requirement changes can come from opportunities enabled by 
creating a new system that can potentially deliver more information to the
driver with faster response times (from the driver) and less distractions. 
This means that any system developed must be flexible enough to allow for easy 
experimenting with configuration
alternatives and have a high-level support for requirements evolution.

The question we want to address in this report is as follows; How can we
enable the development of a state of the art computerized driver information system
by creating a software architecture that is flexible enough to allow easy
experimenting and customization?

The rest of this document is divided into several parts. In the background chapter we will
discuss the collaboration with Revolve NTNU and present the Formula Student
competition. We also present relevant electronic principles and knowledge and discuss the current
state of research with a focus on automotive information systems. Chapter \vref{chapter:method}
introduces the methods we've applied in our work, while
the following chapter (\vref{chapter:results}) shows the findings of our research.
Then we discuss our findings and try to put them in perspective in chapter \vref{chapter:discussion}, before we move
onto the conclusion and discussing recommendations and possibilities for
further work in chapter \vref{chapter:conclusion}.
The appendices contain development artifacts created during the design process
and details the results and the information they build upon, as well as an interview with a driver from last years Revolve-team. 


\footnotetext{
	Although extensive research has been performed scientific, 
	findings are rare. Cottle et al.. \cite{wirelessDashboard} discusses implementing a 
	wireless ``dashboard'' (controls and instruments on a steering wheel). 
	Although lacking scientific base the statement should hold; this 
	assumption is made stronger by analyzing motorsport on TV and for 
	pictures of race cars' interiors.
}
