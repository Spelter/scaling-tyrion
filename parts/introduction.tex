% !TEX root=../thesis.tex

\chapter{Introduction}
% chapter Introduction
\label{chapter:introduction}

This thesis discusses the analysis of train delays and the visualization of 
these delays. The project involves developing a prototype map visualization of 
delays based on train routes in Norway. Such features as the ability to scroll 
through a certain time period (which will be selectable), both backwards and 
forwards will be implemented. The prototype enables studying of information
aggregated through a hierarchy of stakeholders, and study the stakeholders need
for different types of information.

%TODO write better.
%As we discuss in \Vref{chapter:background}, there exists some systems which 
%enables the user to plot train information, however none of them covers what 
%this thesis will research.

\section{Research Question and Goals} % (fold)
\label{sec:intro_research_question}
In this section we will introduce the research question and goals used in this
thesis.\\

A system which allows the user to select between different type of information
to be presented, and at the same time the is aware of the stakeholders that are
relevant to the system, can seem complex. By answering the research question
below, we aimed to produce characteristics which will help defining a
stakeholder aware system.

\begin{itemize}
	\item \textbf{What are important characteristics for a stakeholder aware 
	method to aggregate over a rich set of data?}
\end{itemize}

As part of answering the research question, we developed a prototype. Using the
research method we produced goals for the prototype.

\begin{itemize}
	\item Develop a map based prototype for train analysis.
	\item Sub goals:
	\begin{itemize}
		\item Display dashboard with relevant information accordingly to current zoom level.
		\item Aggregate through statistical data according to current level.
		\item Limit the data visualization within selectable time scales.
		\item Select different type of statistical data to display.
		\begin{itemize}
			\item Display information for traffic density.
			\item Display information for Speed restrictions.
			\item Display information for train crossings.
		\end{itemize}
	\end{itemize}
\end{itemize}
% section research_question (end)

\section{Stakeholders} % (fold)  
\label{sec:intro_stakeholders}  
For any given project, there exists several stakeholders. Stakeholders are 
individuals and/or organizations who have an interest or are involved directly 
in the project, whether the interest is involvement in the execution of the 
project, or the interest overlaps with the results of the project. 
For the thesis, stakeholders are either end users, people with a general
interest in the project, or users which represents the domain the prototype is
developed for. Having a number of stakeholders to be aware of, means that there
are many needs to consider. Since the thesis will be aware of the stakeholders,
the prototype will focus on the stakeholders which represent the domain.
% section stakeholders (end)

\section{Aggregation} % (fold)
\label{sec:intro_aggregation}
When taking into consideration that many different stakeholders demands 
different levels of information presented, the presentation quickly have the 
possibility to become very messy and difficult to understand. One solution to 
avoid too much information being displayed, is to aggregate the data according 
to some kind of sorting of the stakeholders. When one aggregates data, groups 
of observations are replaces with summary statistics based on those 
observations\cite{ wiki:Aggregation}.
% section aggregation (end)

\section{Data Sets} % (fold)
\label{sec:intro_data_sets}
By having many stakeholders to be aware of, along with several types of data,
one of the problems that arises is that the data sets can grow 
quite large. Depending on what one might need, these large data sets can be 
quite problematic to access in a quick and reliable way. This problem grows in 
complexity quickly when one is supposed to take into consideration several 
sets of data containing different type of information. 
% section data_sets (end)

% chapter Introduction (end)
