% !TEX root=../thesis.tex

\chapter{Introduction}
% chapter Introduction
\label{chapter:introduction}

%This thesis discusses the analysis of train delays and the visualization of these delays. The project involves developing a prototype map visualization of delays based on train routes in Norway. Such features as the ability to scroll through a certain time period (which will be selectable), both backwards and forwards will be implemented. The prototype enables studying of information aggregated through a hierarchy of stakeholders, and study the stakeholders need for different types of information.

In this chapter we introduce the problem which we discuss in this thesis.
%TODO
We will first present the research question which we will answer with the help
of the method presented in \Ref{cha:research_questions_and_method} \nameref{cha:research_questions_and_method}.

\section{Background and motivation} % (fold)
\label{sec:background_and_motivation}
In a railway network, trains are run almost constantly to meet the demand for 
passenger and freight transport. The companies responsible for the 
transportation strives for more efficient capacity usage and increased 
punctuality. The Norwegian National Rail Administration (Jernbaneverket 
\cite{jernbaneverketAbout}) collects data about every train driven in the 
railway network and external events affecting the railway. 
Based on the analysis of the collected data, users throughout the companies 
can improve the performance of the infrastructure and the usage of the 
railway.\\

Most of the data collected is kept in different sets throughout the different
companies with a non-coherent definition of data-fields. The structuring of 
the datasets makes comparison difficult for different users. Different users
throughout the companies have different needs when, some have
the need for a large geographical area and a low level of detail resulting in a
overview, while others need a small geographical area and a high level of 
detail. \\


% section background_and_motivation (end)

\section{Research Question and Goals} % (fold)
\label{sec:intro_research_question}
In this section we introduce the research question and goals which we 
answer in this thesis.\\

A system which allows the user to select between different type of information
to be presented, and at the same time to be aware of the stakeholders 
involved, is complex. By answering the research question
below, we aimed to produce characteristics to help define a
stakeholder aware system.\\

\begin{itemize}
	\item \textbf{What are important characteristics for a stakeholder aware 
	method to aggregate over a rich set of data?}
\end{itemize}

As part of answering the research question, we developed a prototype. Using the
research method we produced the following goals for the prototype.

\begin{itemize}
	\item Develop a map based prototype for train analysis.
	\item Sub goals:
	\begin{itemize}
		\item Display dashboard with relevant information accordingly to 
		current zoom level.
		\item Aggregate through statistical data according to current level.
		\item Limit the data visualization within selectable time scales.
		\item Select different type of statistical data to display.
		\begin{itemize}
			\item Display information for traffic density.
			\item Display information for Speed restrictions.
			\item Display information for train crossings.
		\end{itemize}
	\end{itemize}
\end{itemize}

As part of discussing a method for "awareness in presentation", a presentation
of the key parts in the research question is needed. 

By presenting a stakeholder aware method, there are some concerns which needs
to be introduced first. In our problem, we define a stakeholder as a person
within a organizational structure. 

- introdusere problem
- forklare stakeholder problematikk
- rich data set = flere data typer
- prosessere ifølge stakeholder
% section research_question (end)


\section{Stakeholders} % (fold)  
\label{sec:intro_stakeholders}  
As with any project, we had several stakeholders to take into consideration
during the work of this project. A stakeholder is anyone who has a stake in 
the success of the system, which typically have different specific concerns 
that they wish the system to guarantee or optimize 
\cite{Bass:2012:SAP:2392670}.\\

%By presenting a stakeholder aware system, we need to define stakeholders withing the operating domain of the system. 
Stakeholders need to be defined further then just for instance the developers
or project managers, in order to have awareness in how the data is processed 
and presented to the different stakeholders. 
By defining the domain and environment the system will
operate within, further defining of the stakeholders the system will process 
is needed within the domain. When the stakeholders have been defined within 
the domain, their needs and requirements can be defined in a way that the data 
processing methods can take into consideration. \\

When defined within the domain, the stakeholders needs to be organized on a 
structure which enables awareness from the system. During the work of this 
thesis, we present a structure where the stakeholders have been placed in a 
hierarchy according to their areas of responsibility. By organizing a 
hierarchy of the stakeholder, we are able to process the data and limit the 
presentation according to the active stakeholder. 
% section stakeholders (end)

\section{Datasets} % (fold)
\label{sec:intro_data_sets}
Depending on the system and the stakeholders requirements, the amount of data
to process can quickly grow quite large. As the data required by the
stakeholders involves different types of data, the sets often originates from
different sources. In order to allow the system to process the data according
to the stakeholders, the sets need to be defined in a common structure.
Different companies can for instance have different definitions of the same
data field.
% section data_sets (end)


\section{Aggregation} % (fold)
\label{sec:intro_aggregation}
When the stakeholders have been organized in a data processing friendly
structure as presented in \Vref{sec:intro_stakeholders}, we then aggregate the 
data accordingly. To aggregate data means to replace groups of observations
with summary statistics based on these observations \cite{ wiki:Aggregation}.
By aggregating the data according to the stakeholders, we summarize the data
according to the areas of responsibility. The summarization method used on the 
data are specified by the requirements of the stakeholders. We also present
aggregation through time. Each stakeholder are able to limit the data 
according the level of detail needed, by enabling time aggregation.
\\ 
\\ 
\\ 
\\



The work in this thesis was performed in collaboration with the SINTEF
project PRESIS \cite{sintefPresis}. The project is 
a collaboration between SINTEF\cite{sintef}, Transportøkonomisk Institutt
\cite{transportOkonomiskInstitutt}, NTNU\cite{ntnu}, Jernbaneverket(
\Ref{sub:subsection_jernbaneverket}) and the train operators Cargonet (
\Ref{sub:subsection_cargonet}), NSB\cite{nsbForside}, and Flytoget
\cite{flytoget}. The aim of the project is to systematically improve the 
precision level in the railway system by developing methods, tools, and 
processes.

% section aggregation (end)

% chapter Introduction (end)
