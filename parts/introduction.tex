% !TEX root=../thesis.tex

\chapter{Introduction}
% chapter Introduction
\label{chapter:introduction}

This thesis will discuss analysis of train delays and visualization of these
delays. It will involve making a prototype map visualization of delays based on
train routes in Norway. This prototype will involve features such as being able
to play through a certain time period, which will be selectable, both backwards
and forwards. It will also make it possible to study each train delay to make
it possible to track the delays through the Norwegian rail network, and
hopefully make it possible to spot why a train is delayed.

We discuss this in Chapter \vref{chapter:background}, some plotting 
examples exists already, however none of them covers what this thesis will 
research.

\section{Research Question and Goals} % (fold)
\label{sec:intro_research_question}
\begin{itemize}
	\item What are important characteristics for a stakeholder aware method to 
	aggregate over a rich set data?
\end{itemize}

Goals:

\begin{itemize}
	\item Develop a map prototype.
\end{itemize}

Sub goals:

\begin{itemize}
	\item Display dashboard with relevant information accordingly to current zoom level.
	\item Aggregate through statistical data according to current level.
	\item Select different type of statistical data to display.
\end{itemize}
% section research_question (end)

\clearpage
\section{Stakeholders} % (fold)  
\label{sec:intro_stakeholders}  
For any given project, there exists
several stakeholders. A stakeholder are individuals and organizations who have a interest in the
project, whether it be involvement in the execution of the project, or have interest which overlap
with the results of the project. Because each stakeholder either represents different end user 
types or people with a general interest in the project, in the end there will be many needs to 
take into consideration. Since this will eventually turn into a stakeholder aware system, with 
focus on stakeholders representing end users, there will end up presenting a lot of data. 
% section stakeholders (end)

\section{Aggregation} % (fold)
\label{sec:intro_aggregation}
When taking into consideration that many different stakeholders demands different levels of
information presented, the presentation quickly have the possibility to become very messy and
difficult to understand. One solution to this, is to aggregate the data according to some kind of
sorting of the stakeholders. When one aggregates data, groups of observations are replaces with
summary statistics based on those observations\cite{wikiAggregation}.
% section aggregation (end)

\section{Data Sets} % (fold)
\label{sec:intro_data_sets}
One of the problems with a task such as this, is that the data sets can grow quite large. Depending
on what one might need, these large data sets can be quite problematic to access in a quick and
reliable way. This problem grows in complexity quickly when one is supposed to take into
consideration several sets of data containing different type of information. 
% section data_sets (end)

% chapter Introduction (end)
