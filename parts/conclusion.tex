% !TEX root=../thesis.tex
\chapter{Conclusion and future work}
\label{chapter:conclusion}
	
In this report we have discussed our proposed driver information system,
ReVision, and its proposed architecture. The project is being developed in
collaboration with Revolve NTNU, which is a team of volunteering students
participating in the Formula Student series of competitions with their
self-designed and built car. The idea behind the system is to create an
enabling driver information system for use in motorsport -- an area where
driver information systems have not developed much, despite great developments
in other areas. 

We propose to change that by developing a computerized driver information
system that employs a head-up display to convey information to the driver. Due
to restrictions in the the rules and regulations of the FS events, the system
is best realized as a wearable display system, eg. instance using spectacles with a see-through screen. 

The software architecture we have designed is designed with the purpose of
creating a system that allows us to experiment with layouts and design in an
easy way. This way we can experiment with what kinds of information is
displayed, and the visualization-style. The architecture also
needs to support distribution over multiple nodes and performing concurrent
sequential transformations of raw input data, through real-world values, and on
to graphical representations of that data. 

The proposed architecture has not been verified and to that point our
suggestions for an architecture design can only be considered qualified
guesswork. Although we have designed the architecture based on patterns and ideas thought up by people with more experience and knowledge than us -- we do not yet know if that architecture will be enabling us in reaching the
system goals and requirements. 

Our conclusion is that the system described herein has the potential to give a
motorsport performers a significantly better way of acquiring information. We
also have a hypothesis that the benefits might be biggest during practice and training;
where the system could help drivers actively get feedback on their driving
instead of having to stop and review telemetry data outside of the car. 

\section{Further work}

In a further development process we find that it
would be natural to build upon the findings from this project; develop
further metrics to measure the systems ability to meet goals, and to refine
both goals and requirements as time passes. Changes in use cases and design
should follow naturally with further development.

Taking the system further requires a methodical approach to performance testing
and evaluation. It is imperative that any performance bottlenecks are found as
early as possible. We also suggest the possibility of using performance 
modeling tools to assess
performance when a baseline performance of the hardware architecture has been
found. 

We believe that the proposed system has the potential of contributing to
motorsport and that the project deserves to be realized. There are however some
high level risks with regards to acquiring the necessary hardware to realize
the HUD-portion of the system which should be researched more before further
work is started. 
