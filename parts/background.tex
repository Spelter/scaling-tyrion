% !TEX root=../thesis.tex

\chapter{Background}
\label{chapter:background}

A train network is a complex system. Almost every running train have the 
possibility to affect almost every other train running in the system.  When 
you look at a busy area, such as a major city and it's closest
area, a great deal of trains can be on the move at any given time on a
railway network with limited capacity. This leads to limited time slots for each 
train and every problem can lead to major delays.


Even though one train may be experiencing delay, this delay may be part of a
sequence of problems that can be tracked back to a seemingly unrelated part of
the the network and a perhaps a bad decision there\cite{cule2011mining}. \\

In the Norwegian railroad a train is on schedule if it arrives the final
destination within a margin on 3 minutes and 59 seconds, if it is a long
distance train the margin is 5 minutes and 59 seconds. 
Jernbaneverket (see\vref{sub:subsection_jernbaneverket}) defines regularity as the number of trains that gets run as 
planed in the time schedule. 
Uptime in regards to punctuality is defined by Jernbaneverket from the hours of delay\footnote{Hours of delay due to infrastructure excluded traffic	management and external conditions} caused by infrastructure relative to sum of planed train hours\footnote{Planed train hours (passenger and freight trains)} per year.
\cite{jernbaneverketPunklighetsTall}
\begin{equation} Uptime =
		\frac
				{
					\text{Train hours - Hours of delay}
				}
				{
					\text{Train hours}
				}\times 100 
\end{equation}\\

As Landex\cite{landex2009gis} says, there exist few GIS-approaches concerning
visualization of railroad capacity. Both the visualizations shown by Landex and
in section \vref{sect:backgroundExamples} only seems to take into consideration if
the trains are delayed, and the amount of delay. 

However, to minimize the delays all over the railway network, it may be necessary
to not only see that certain routes are delayed but also why it is delayed. To
be able to understand this, you need to be able to mine data from the railway
network administrator and have a good visualization tool to present it. 

\pagebreak
% !TEX root=../../thesis.tex
\clearpage
\section{Case study of similiar systems}
\label{sect:backgroundExamples}
This section will present some similar systems and we will explain what the
systems tries to present.
\subsection{Zugmonitor}
\label{sub:subsection_zugmonitor}

In the Zugmonitor application (see \Ref{fig:zugmonitor}) each long-distance 
train in the German railway network has been plotted as a arrow on a German 
map. To illustrate the punctuality of each train, a colored circle has been 
added to each arrow if the train is delayed with varying color depending on 
how big the delay is. A time line is also displayed to see how the trains run 
on each step of the routes. 

\begin{figure}[!htbp]
	\includegraphics[width=0.7\textwidth,center]{zugmonitor.png}
	\caption[Zugmonitor]{Zugmonitor \cite{zugmonitor}}
	\label{fig:zugmonitor}
\end{figure}

\subsection{Vaguely live map of trains in the United Kingdom}
\label{sub:subsection_ukLiveMap}

The "Vaguely live map" system is a map which plots the relative location of 
each train in the United Kingdom (see \Ref{fig:ukLiveMap}). The plot fetches 
the departure time from the  National Rail website and calculates the relative 
location. The plot does not indicate whether the trains are on schedule or 
delayed, if one wants to check for delayes, one either has to do it manually 
on station, or for instance by checking a time table\cite{trainTimesUK}. Both 
the map and time table is developed on hobby basis by the same person. 

\begin{figure}[!htbp]
	\includegraphics[width=0.7\textwidth,center]{live-train-map-for-Birminingham-new-street.png}
	\caption[UK live map]{UK live map \cite{ukLiveMap}}
	\label{fig:ukLiveMap}
\end{figure}

\subsection{MUNI Light Rail}
\label{sub:subsection_muniLightRail}

The "Muni light rail" system is a train graph based on the N-Judah line on the 
Muni Metro light railway line in San Francisco (see \Ref{fig:muniLightRail}). 
The train graph plots the schedule of the each train and the actual time each 
train uses. The chart auto updates every 10 seconds, and combined with being 
able to spot the difference between the schedule and the actual time, makes it 
easy to spot the delay of each train. As with \Ref{sub:subsection_ukLiveMap}
the "Muni light rail" system has been developed on a hobby basis.

\begin{figure}[!htbp]
	\includegraphics[width=0.7\textwidth,center]{visualizing-transit-delays.png}
	\caption[MUNI Light Rail]{MUNI Light Rail \cite{muniLightRail}}
	\label{fig:muniLightRail}
\end{figure}

\subsection{MiseryMap}
\label{sub:subsection_zugmonitor}

The MiseryMap (see \Ref{fig:miserymap}) shows how much different airports and 
the routes between them are delayed. The system also have a playback function 
to see the delays throughout the day. Since the plot sometimes also shows the 
weather conditions, the system gives the user the possibility to spot if the 
delays can be blamed on uncontrollable conditions. 

\begin{figure}[!htbp]
	\includegraphics[width=0.8\textwidth,center]{MiseryMap.png}
	\caption[MiseryMap]{MiseryMap \cite{flightAware:MiseryMap}}
	\label{fig:miserymap}
\end{figure}
 
\subsection{Norwegian National Rail Administration}
\label{sub:subsection_jernbaneverket}

Jernbaneverket is the Norwegian governments agency for railway services 
\cite{jernbaneverketAbout}. As described in \Ref{sec:railway_operations}, 
Jernbaneverket is responsible for the infrastructure of all the railway 
network in Norway. Since they have responsibility for the infrastructure, 
they also collect data from points that each train passes, as described in 
\Ref{sec:back_data_sets}. Based on the collection of data, they have recently 
released a map (see \Ref{fig:jernbaneverket-punklighet}) over the punctuality 
on each track segment. The released map is a interactive map which shows a 
pop-up box containing the punctuality of the train segment clicked on, and the 
pop-up also shows which train routes that operates the train segment clicked 
on. The map, does not however, show more information if the user zooms inn, 
which is possible within the map itself, and has a static view of Norway and 
the railway. The map only shows the punctuality for passenger trains, and not 
freight trains and/or both.\\ 

To analyze each stretch, on a detailed level between each station,
Jernbaneverket has a internal system called TIOS.
TIOS can create a train graph which plots all trains that passes between all 
stations, \Ref{fig:jernbaneverket-tios} presents an example of the train graph 
between Oslo S and Drammen, where the red lines means planned time, black 
lines is actual time and the red circle indicates a a code for the cause of 
the delay. 

\begin{figure}[!htbp]
	\includegraphics[height=0.4\textheight]{jernbaneverket-punklighet.png}
	\caption[JBV Punctuality map]{JBV Punctuality map \cite{jernbaneverketPunklighetKart}}
	\label{fig:jernbaneverket-punklighet}
\end{figure}
\begin{figure}[!htbp]
	\includegraphics[width=\textwidth]{tios.png}
	\caption[TIOS]{TIOS\cite{jernbaneverketAbout}}
	\label{fig:jernbaneverket-tios}
\end{figure}

\subsection{Tåg.info}
\label{sub:subsection_taag.info}

Tåg.info\cite{taagInfo} is a Swedish system that tracks SJ\cite{svenskaJernban}
trains. The service gathers and processes data from Trafikverket\cite{trafikverket}. As with Cargonet (\Ref{sub:subsection_cargonet}),
Tåg.info provides a method (\Ref{fig:taag-info-kart}) of tracking live trains and visually see whether the trains are on schedule or delayed. 

Tåg.info also provides a method of analyzing national delays by presenting
graphs, \Ref{fig:taag-info-historik}. The service presents both a bar-chart
which presents the minutes of delays and the accumulated delays per day, and a
pie chart of the current status.

\begin{figure}[!htbp]
	\includegraphics[width=0.7\textwidth,center]{taag-info-kart.png}
	\caption[Tåg.info map]{Tåg.info map
	\cite{taagInfo}}
	\label{fig:taag-info-kart}
\end{figure}

\begin{figure}[!htbp]
	\includegraphics[width=0.7\textwidth,center]{taag-info-historik.png}
	\caption[Tåg.info history]{Tåg.info history
	\cite{taagInfo}}
	\label{fig:taag-info-historik}
\end{figure}


\subsection{SINTEF Presis}
\label{sub:subsection_sintefPresis}

The PRESIS\cite{sintefPresis} project is a collaboration between SINTEF\cite{sintef},
Transportøkonomisk Institutt\cite{transportOkonomiskInstitutt},
NTNU\cite{ntnu}, Jernbaneverket(\Ref{sub:subsection_jernbaneverket}) 
and the train operators Cargonet (\Ref{sub:subsection_cargonet}), NSB\cite{nsbForside}, and Flytoget\cite{flytoget}. The aim is to systematically improve the precision 
level in the railway system by developing methods, tools, and processes. During
the PRESIS project several prototypes for analyzing train delays have been 
developed.

A interaction plot, see \Ref{fig:krysningsinteraksjon}, plots the interaction
between two selected trains at a selected station. The interaction plot makes 
it easy to see if a train is delayed, and how the delayed train might affect 
another train. Since the interaction plot only plots the interaction between 
two trains, the plot makes it difficult to follow a delay back through the 
railway network to find the source. Due to not being able to track the delay
backwards, the plot is useful for individual trains, but are difficult to use
if one wants to look at the entire railway network or a large portion of it. \\

Publicly and internally the status of a train may differ whether a train are 
delayed or not. Public statistical data only shows delays on the final 
destination of the train, however all stations along the route show whether 
the train are on schedule or not on information screens on each station. Since 
Jernbaneverket collects data from all signal points along the route (see 
\Ref{fig:jernbaneverket-trafikkdata}), Jernbaneverket are able to track and 
plot the delays the train may experience along the route and not only at the 
end station. The PRESIS project has developed a prototype for such a plot, see 
\Ref{fig:live-punklighet}. Here the circles represent a station and the lines 
between the station have different colors which represent the punctuality 
between the stations, and the data which the plot is based on are listed to 
the right. The PRESIS project have also made it possible to get a time used 
over distance plot (see \Ref{fig:plot-spc-for-strekning}), based on the 
selected data in the Punctuality for routes plot. 

A plot based on time over distance, \Ref{fig:plot-spc-for-strekning}, plots the
actual time used by all trains that have driven that stretch in the selected
time period along with the running average. When plotting based on time over 
distance, the plot enables the possibility to spot where and
when trains have experienced problems on the selected stretch. They also have a
prototype plot which plots time used on station (see \Ref{fig:plot-spc-for-stasjonsopphold})
and other prototypes which is similar, just focusing on other parts of the
process.  %Bidragsplot, ikke tatt med.


%\begin{figure}[!htbp]
%	\centering
%	\begin{subfigure}{0.3\textheight}
%		\includegraphics[width=0.3\textheight,center]{krysningsinteraksjon.png}
%		\caption[Train interaction plot]{Train interaction plot \cite{sintefPresis}}
%		\label{fig:krysningsinteraksjon}
%	\end{subfigure}
%	\begin{subfigure}{0.4\textheight}
%		\includegraphics[height=0.4\textheight,center]{live-punklighet.png}
%		\caption[Route punctuality]{Route punctuality\cite{sintefPresis}}
%		\label{fig:live-punklighet}
%	\end{subfigure}
%	\caption[JBV Punctuality map and TIOS]{JBV Punctuality map and TIOS}
%	\label{fig:nbaneverket-punklighet_and_jernbaneverket-tios}
%\end{figure}

\begin{figure}[!htbp]
	\includegraphics[width=\textwidth,center]{krysningsinteraksjon.png}
	\caption[Train interaction plot]{Train interaction plot \cite{sintefPresis}}
	\label{fig:krysningsinteraksjon}
\end{figure}

\begin{figure}[!htbp]
	\includegraphics[height=\textheight,center]{live-punklighet.png}
	\caption[Route punctuality]{Route punctuality\cite{sintefPresis}}
	\label{fig:live-punklighet}
\end{figure}

\begin{figure}[!htbp]
	\includegraphics[width=\textwidth,center]{plot-spc-for-strekning.png}
	\caption[SPC Stretch]{SPC Stretch \cite{sintefPresis}}
	\label{fig:plot-spc-for-strekning}
\end{figure}

\begin{figure}[!htbp]
	\includegraphics[width=\textwidth,center]{plot-spc-stasjonsopphold.png}
	\caption[SPC Station]{SPC Station \cite{sintefPresis}}
	\label{fig:plot-spc-for-stasjonsopphold}
\end{figure}

\begin{figure}[!htbp]
	\includegraphics[width=\textwidth,center]{ukespunklighet.png}
	\caption[Weekly punctuality]{Weekly punctuality\cite{sintefPresis}}
	\label{fig:ukespunklighet}
\end{figure}

\subsection{Cargonet} % (fold)
\label{sub:subsection_cargonet}

% subsection subsection_sintefPresis (end)
Cargonet is a Norwegian company which provides intermodal transport on rails. 
To provide an effective tracking service for the customers, Cargonet provides 
an internal service for the users which tracks all trains belonging to Cargonet.
As can be seen in \Ref{fig:cargonet}, the system only shows a picture of the 
current status of each train. The system lacks the possibility to analyze 
every stretch individually and analyze trains and stretches over time.

\begin{figure}[!htbp]
	\includegraphics[width=\textwidth,center]{cargonet.png}
	\caption[Cargonet]{Cargonet \cite{cargonet}}
	\label{fig:cargonet}
\end{figure}

\begin{itemize}
	\item [] \textbf{How to read \Ref{fig:cargonet}}
	\item Red arrow:\hspace{4ex} Delayed.
	\item White arrow:\hspace{4ex} On time.
	\item Red box:\hspace{4ex} Locomotive driven 2km without carriages.
	\item Black box:\hspace{4ex} Locomotive without carriages.
	\item Yellow box:\hspace{4ex} Locomotive on time without schedule, or known position.
\end{itemize}

% subsection subsection_cargonet (end)

% !TEX root=../../thesis.tex

\section{Defining frameworks} % (fold)
\label{sec:defining_frameworks}
This section will first define the users in the railway network, and then
define framework(s) based on the need of the users and what the examples
presented in section \vref{sect:backgroundExamples} tries to present.

\subsection{User roles in the Norwegian railway network} % (fold)
\label{sub:user_roles_in_the_norwegian_railway_network}

Combined in all the companies with a certain interest in the Norwegian railway 
network there are several types of users which have different perspective of
the railway. 
\begin{table}[!h]\small
	\begin{tabularx}{\textwidth}{|l|l|X|}
		\hline
		Company & User type & Responsibility \\
		\hline
		Jernbaneverket & Traffic director & Responsibility to facilitate that the railway traffic in Norway is safe and reliable\\
		\hline
		Jernbaneverket & Area director & Responsibility to facilitate that the railway traffic in Norway is safe and reliable\\
		\hline
		Jernbaneverket & Segment director & Responsibility to facilitate that the railway traffic in Norway is safe and reliable\\
		\hline
		NSB & Nation responsible & 1 per line (Dovre, Bergen, etc.)\\
		\hline
		NSB & East responsible & 1 per defined area in the east of Norway (Vestfold county, Oslo - Hamar,
		etc.)\\
		\hline
		None & Average Joe & General user interested in train punctuality\\
		\hline
	\end{tabularx}
\caption{User roles in the Norwegian railway network}
\label{table:user_roles}
\end{table}

Based on the perspective of these users, different information is useful and
not all examples in section \ref{sect:backgroundExamples} may be useful. They
may have just a basic interest whether trains are delayed or not, for instance
traveling with trains on holiday; or have a more detailed need to understand
all train delays in the whole network, for instance to make new schedules. 

% subsection user_roles_in_the_norwegian_railway_network (end)

\clearpage
\subsection{Information presented vs information needed} % (fold)
\label{sub:information_presented_vs_information_needed}

Since each project presented in section \vref{sect:backgroundExamples} is
based different type of user accessing it, they present different amount of
information. Since some the projects presented contains different types of
information presentations and they may cover different needs, the following 
definition will be based on the presented figures. 

\begin{table}[!h]\small
	\begin{tabularx}{\textwidth}{|l|l|X|}
		\hline
		User & Need & Figure \\
		\hline
		Traffic director & Overview/detailed for the nation & Responsibility to facilitate that the railway traffic in Norway is safe and reliable\\
		\hline
		Area director & Overview/detailed for the area & Responsibility to facilitate that the railway traffic in Norway is safe and reliable\\
		\hline
		Segment director & Detailed for each segment & Responsibility to facilitate that the railway traffic in Norway is safe and reliable\\
		\hline
		Nation responsible & Detailed for each line & 1 per line (Dovre, Bergen, etc.)\\
		\hline
		East responsible & Detailed for east & 
						\vref{fig:muniLightRail} \newline
						\vref{fig:jernbaneverket-tios} \newline
						\vref{fig:krysningsinteraksjon} \newline
						\vref{fig:plot-spc-for-strekning} \newline
						\vref{fig:plot-spc-for-stasjonsopphold} \newline
						\vref{fig:ukespunklighet}\\
		\hline
		None & Basic & 	\vref{fig:zugmonitor}\newline
						\vref{fig:ukLiveMap} \newline
						\vref{fig:miserymap} \newline
						\vref{fig:jernbaneverket-punklighet}\\
		\hline
	\end{tabularx}
\caption{Information presented vs information needed}
\label{table:information_presented_vs_information_needed}
\end{table}


% subsection information_plotted_vs_information_needed (end)

% section defining_frameworks (end)

