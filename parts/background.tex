% !TEX root=../thesis.tex

\chapter{Background}


%Piloting map service for navigating in punctuality analysis for trains

A train network is a complex system. Almost every running train have the 
possibility to affect almost every other train running in the system.  When you look at a busy area, such as a major city and it's closest
area, a great deal of material can be on the move at any given time on a
rail network with limited capacity. This leads to limited time slots for each train and every
problem can lead to major problems, not just for the train experiencing the
problem, but can spread to other trains. \\

To minimize delays it may be necessary to improve both infrastructure and/or
time table on railway routes or parts of routes. However, to understand what
needs to be improved and optionally where, you need a good tool to analyze the
rail network capacity, and if necessary, visualize each individual train (see 
\vref{fig:zugmonitor}) in the rail network to follow delays to the source.

Here they have plotted each train on it's course between each station on it's
route, with a colored circle around the train which varies depending on if the
train is on schedule or delayed. It is also possible to play through the
selected day or drag through time manually.\\

\begin{figure}[!htbp]
	\includegraphics[width=\textwidth,center]{zugmonitor.png}
	\caption[Zugmonitor]{Zugmonitor \cite{zugmonitor}}
	\label{fig:zugmonitor}
\end{figure}

In the next example (\vref{fig:ukLiveMap}) a live map for the train 
routes in the United Kingdom have been developed, where it plots the relative 
location of trains in the UK based on live departure data fetched from the 
National Rail website. This example does not however, display whether or not 
the trains are delayed. The routes are only drawn as straight lines between each station.\\

\begin{figure}[!htbp]
	\includegraphics[width=\textwidth,center]{live-train-map-for-Birminingham-new-street.png}
	\caption[Vaguely live map of trains in the United Kingdom]{Vaguely live map of trains in the United Kingdom \cite{ukLiveMap}}
	\label{fig:ukLiveMap}
\end{figure}

The third example (\vref{fig:muniLightRail}) displays a XY-chart based on the 
N-Judah line on the Muni Metro light rail line in San Francisco, with stations 
on the Y-line and time on the X-line. This chart plots the schedule of the each
train and the actual time each train uses. \\

\begin{figure}[!htbp]
	\includegraphics[width=\textwidth,center]{visualizing-transit-delays.png}
	\caption[Visualizing transit delays]{Visualizing transit delays \cite{muniLightRail}}
	\label{fig:muniLightRail}
\end{figure}

The next example (\vref{fig:miserymap}) shows how much different airports and 
the routes between them are delayed. It also have a playback function to see
how the delays are throughout the day. This plot also shows some weather so it
may be possible to spot if the delays to be blamed on uncontrollable
conditions. 

\begin{figure}[!htbp]
	\includegraphics[width=\textwidth,center]{MiseryMap.png}
	\caption[MiseryMap]{MiseryMap \cite{flightAware:MiseryMap}}
	\label{fig:miserymap}
\end{figure}
 
