% !TEX root=../../thesis.tex

\section{Characteristics} % (fold)
\label{sec:characteristics}
This will mainly be melted inn conclusion and results. Was meant for typing out
list in \Ref{sec:list_learned_to_be_removed}.

\textbf{Research question}
\begin{itemize}
	\item What are important characteristics for a stakeholder aware method to 
	aggregate over a rich set of data?
\end{itemize}

When one needs a method that are aware of stakeholders for aggregating, it is
important to have a clearly defined hierarchy of the stakeholders and their 
responsibility area. When implementing the stakeholder hierarchy in the
developed prototype, it was implemented as a selectable list in the upper right
corner, see \Ref{fig:stakeholder_selection_list}. This means that when 
one wants to move in the hierarchy, one simply clicks their way through the list to the wanted area.


Since by navigating through the stakeholder hierarchy by selecting different
levels in the selectable list means that the aggregation logic in the 
background determines what is displayed, it provides a good example of
displaying information based on what fits the needs of the current stakeholder.
When the display receives the return data, it fits the view of the map to
data. When sending queries the map also sends along time data, from user 
inputs. This time data, is taking into consideration with the stakeholder when 
determining what kind of data should be returned and displayed.\\

Since the data is stored in sets, it is possible to do a intersect on the data
based on the requirements of the stakeholder and the selectable time data.
By combining these two, one is able to navigate through the  data both in time 
and space and it is possible to limit the information displayed at any given
time based on the needs of the stakeholder.\\

%XXX visualiser med Venn-diagram\\

Since the methods that shall aggregate over the data sets have to to consider
both the stakeholder and the data type, the methods should be dynamic so that
they can be reused throughout every stakeholder and every type of data. To 
achieve this dynamic, one should have loosely linked data with stakeholders
such as that the data sets are easily editable in terms of adding or removing
data, or adding new type of data sets.

Since most of the statistical data used in this thesis, are sorted on one 
identifier and that identifier is connected to the stakeholders, there are 
quite easy to expand the data sets. This means that the aggregation is actually
a three step process.

\begin{description}
	\item [1. Identify stakeholders needs] By searching the set of data that 
	have the connection to the stakeholders based on a given stakeholder, one 
	is able to generate a list of all the identifiers relevant to this 
	stakeholder.
	\item [2. Retrieve relevant data] By using the the list of relevant
	identifiers generated based on the stakeholder, one is able to find all 
	wanted data.
	\item [3. Aggregate through the found data] When one have acquired the 
	wanted data, one aggregates through the data according to the 
	specification of that set of data.
\end{description}

As described by Bass, Clements, and Kazman
\cite[pp. 117-124]{Bass:2012:SAP:2392670}, most changes to a system happens 
after initial release. This means that by storing the data sets in 
independently structures, one can have loose coupling between the structures. 
This also means that when the data sets are designed with built-in 
flexibility, then exercising this flexibility is usually cheaper than to hand-
code a specific change in hindsight. By having these loosely linked data sets,
one have achieved a modifiable data storage. This thesis demonstrated this
principal by having the stations stored in a hierarchy data set based one the 
stakeholder hierarchy, and the other data sets connected to the stations.\\

By having a loosely linked storage of the different data sets, an important
thing to take into consideration is optimization of Step 2 listed above. Since
one is able to navigate through time as described, the system is able to limit
the amount of data it has to process at each execution. By adding clauses which
will limit the result, the query time will drop and the performance time of the
aggregation will improve, both in terms of waiting for returning data from
queries on the data sets and with processing said data.\\

By having the user selecting the wanted stakeholder and what type of
information the system shall present, it enables the system dynamically
determine which data set it shall look in and aggregate through. This is
demonstrated in \Ref{fig:stakeholder_selection_list} which is the
selection of stakeholders and in \Ref{fig:implementation_type_selection} is the selection of data
sets.

Since the aggregation methods by this point have received wanted stakeholders,
the wanted type of information to be display according to the stakeholders
needs, and time interval for limiting the data set, the aggregation should now
be able to generate the correct data according to the stakeholders
requirements. The methods should then, according to step 2 retrieve the data
and loop through it according to the requirements. As was demonstrated in this 
prototype, most data was a simple count of occurrences. This meant then when
aggregation through the responsible areas of each stakeholder it was a mostly
done by averaging all occurrences beneath that area and generating a dummy
marker for each sub-area with the aggregated data of that area.


% section characteristics (end)
