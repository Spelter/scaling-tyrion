% !TEX root=../../thesis.tex

\section{Workshop 2014-04-04} % (fold)
\label{sec:workshop_2014_04_04}
The first workshops agenda was to help define the stakeholders of the system,
their areas of responsibility, and their needs. The workshop was also meant to 
bring clarity to how the system will relate to these users and their needs.
Attending this workshop was Andreas Amdal Seim (SINTEF), Andreas Dypvik 
Landmark (SINTEF), Rimmert van der Kooij (SINTEF), Nils Olsson (NTNU), Per 
Magnus Hegglund (Jernbaneverket), Magnus Bae (NTNU), and Magnus Krane (NTNU).\\

Perspectives that can be used when viewing the system were discussed
first.
\begin{itemize}
	\item Infrastructur: Segment director, etc.
	\item Traffic division / passenger / train companies.
	\item Delay causes: delay demographic.
\end{itemize}

Interests of users when using the system were then discussed based on the
perspectives.
\begin{itemize}
	\item Uptime, punctuality.
	\item Deviation.
	\begin{itemize}
		\item What?
		\item Where?
		\item When?
	\end{itemize}
	\item Delay time.
	\item Variation.
	\item Changes.
\end{itemize}

Causes that might affect the punctuality was also listed: weather, number of
passengers, capacity utilization, animal accidents, cargo volume.
The causes was concluded not to be included in this project, as the causes is
difficult to prove and get data on.\\

The internal project in Jernbaneverket (section
\Ref{sub:subsection_jernbaneverket}) uses a deviation registry for data to
analyze each stretch on a detail level. To calculate the uptime, presented in
\Ref{sec:railway_operations}, also uses the deviation registry for the needed
data. A problem by using the deviation registry for calculating variation or 
changes, the registry have a five minute filter in which the trains are being 
calculated to be on time.

Two problems were agreed upon that needed to be addressed, the back end and 
the  front end of the system. What kind of data is available and what is 
possible to do with this data? A data set can show both positive and negative 
results, based on what the set are compared too. For instance, easter is not 
on the same week each year and the passenger volume increase during the 
holidays. Data sets ended up being addressed in the second workshop (\Ref{sec:workshop_2014_04_24}). 

As the different stakeholders might have need for different presentation of the
data, different levels of stakeholders and what should be shown in each level 
was discussed. A suggestion was made to have the same perspective through the
levels, but to have selectable views based on roles. \\

At the end of the workshop, three conclusions was made. The first conclusion 
was to have a dashboard next to each marker with relevant data to the current 
stakeholder. The second conclusion was to a interactive list of the
stakeholders which adapted the visual presentation to the selected stakeholder,
presented in \Ref{sec:back_stakeholders}.
The last conclusion was to have a second workshop where the content of the
dashboard should be decided. This resulted in \Ref{sec:workshop_2014_04_24}.

% section workshop_2014_04_04 (end)

\section{Workshop 2014-04-24} % (fold)
\label{sec:workshop_2014_04_24}
The second workshops agenda was to determine what statistical data was 
to be implemented in the dashboard, concluded upon in \Ref{sec:workshop_2014_04_04}.
Attending this workshop was Andreas Amdal Seim (SINTEF), Andreas Dypvik 
Landmark (SINTEF), Rimmert van der Kooij (SINTEF), and Magnus Krane (NTNU).\\

The workshop started with a brainstorming for different data to present in
the map. The different data was ranked on implementation practicality from 1 - 
3 where 1 is unpractical and 3 is very practical; and ranked on the 
desirability of the data from 1 - 3 where 1 is undesirable and 3 is very 
desirable. The brainstorming after ranking is presented in \Ref{table:dashboard_functionality_wants_vs_needs}.
\\

\begin{table}[!h]\small
	\begin{tabularx}{\textwidth}{|X|l|l|l|}
		\hline
		Functionality & Practicability & Desirable & Priority\\
		\hline
		Outstanding errors & 1 & 1 & 1\\
		\hline
		Suspensions & 3 & 1 & 3\\
		\hline
		Variation & 1 & 3 & 3\\
		\hline
		Season effects & 3 & 1 & 3\\
		\hline
		Follow delays & 1 & 3 & 3\\
		\hline
		Speed limits & 3 & 1 & 3\\
		\hline
		Cause & 3 & 2 & 6\\
		\hline
	 	Worst stretch/station/train number & 3 & 2 & 6\\
		\hline
		Delays & 3 & 3 & 9\\
		\hline
		Traffic density & 3 & 3 & 9\\
		\hline
		Speed restrictions & 3 & 3 & 9\\
		\hline
		Crossings & 3 & 3 & 9\\
		\hline
	\end{tabularx}
\caption{Dashboard functionality brainstorming ideas}
\label{table:dashboard_functionality_wants_vs_needs}
\end{table}

Based on the ranking, the decision was made to only implement the data was
ranked as a 3 both on practicability and desirability, presented in the list
below.

\begin{itemize}
  \item Speed restrictions
  \item Crossings
  \item Traffic density
\end{itemize}

The second part of the agenda, was to determine how the selected data was to be
presented. The decision was made to split the presentation in two parts. The
first was to display aggregated data in the dashboards next to the marker based
on the current stakeholder. The second was to display top 5 lists for delays
and speed restrictions.

\subsection{Crossings} % (fold)
\label{sub:crossings}
One of the problems discussed with crossing was should the system take into 
account the difference between actual crossings and planed crossings or not? 
The difference is hard to calculate as one does not know whether one of the 
trains involved in the planned crossing, was canceled or just delayed. The 
decision was made to present the number of crossings occurred at each station, 
and aggregate the number as one navigates in the stakeholder hierarchy.
%TODO planned crossings, ask Landmark.
% subsection crossings (end)

\subsection{Traffic density} % (fold)
\label{sub:traffic_density}
Several options to present traffic density were discussed, the problem was how
to correctly show the number of trains based on the data available. 
Should the system present each train based on the train numbers and 
calculate for the entire line both ways? Should the system display train 
numbers divided by the segment directors? 

In the workshop it was decided to show the number of trains that passes each 
block segment, based on the data available. When navigating in the hierarchy,
the system shall aggregate the number of trains.

% subsection traffic_density (end)

\subsection{Speed restrictions} % (fold)
\label{sub:speed_restrictions}
Speed restrictions was decided to be presented a marker per restriction, and be
shown between the selected time interval. The data for the speed restrictions
was to be presented in a plot which appeared in the marker for each
restriction.

%In the dashboard it was decided to show the top 5 upper and lower bounds. It was also decided to include a little marker on the map to indicate the location of the speed restriction. 

% subsection speed_restrictions (end)

% section workshop_2014_04_245 (end)

% section workshops (end)
