% !TEX root=../../thesis.tex

\section{Workshops} % (fold)
\label{sec:workshops}
In order to determine what is relevant content for this thesis, workshops were
held. These workshops were meant to bring some context and direction for the
thesis. 

\subsection{Workshop 2014-04-04} % (fold)
\label{sub:workshop_2014_04_04}
The first workshop was meant to bring some clarity to what kind of users this
system will expect to be used by; their responsibility, and their needs.
It was also meant to bring clarity to how the system will relate to these users and their needs.
Attending this workshop was Andreas Amdal Seim (SINTEF), Andreas Dypvik 
Landmark (SINTEF), Rimmert van der Kooij (SINTEF), Nils Olsson (NTNU), Per 
Magnus Hegglund (Jernbaneverket), Magnus Bae (NTNU), and Magnus Krane (NTNU).\\

At first it was discussed that it is 3 sets of "glasses" you can put on when
viewing the system.
\begin{itemize}
	\item Infrastructur: Segment director, etc.
	\item Traffic division / passenger / train companies.
	\item Delay causes: delay demographic.
\end{itemize}

Based on this, it was discussed what the users might be interested in when
using the system.
\begin{itemize}
	\item Uptime, punctuality.
	\item Deviation.
	\begin{itemize}
		\item What?
		\item Where?
		\item When?
	\end{itemize}
	\item Delay time.
	\item Variation.
	\item Changes.
\end{itemize}

Causes that might affect the punctuality was also listed: weather, number of
passengers, capacity utilization, animal accidents, cargo volume.
It was decided that these causes is difficult to prove and get data on, and
therefore was not to be included at this project.\\

The internal project in Jernbaneverket (section
\Ref{sub:subsection_jernbaneverket}), which is being used to analyse each
stretch on a detail level, gets it's data from a deviation registry; also the
uptime uses this registry for it's calculations. The problem here is that the deviation registry has a five minute filter
in which the trains are being calculated to be on time. (XXX)

It was then agreed that it was two problems that had to be addressed, back end
and front end of the system. What kind of data is available and what is
possible to do with this data? It was also pointed on that same data set can 
show both successful and failure results based on when it compared to, for 
instance the easter is not on the same week each year. This ended up being 
addressed in workshop two (\Ref{sub:workshop_2014_04_24}). 

In the front end it was discussed different levels and what should be shown in
each level, that different users might have different needs in each level then
the other users. It was suggested to have the same perspective through the
levels, but too have selectable views based on roles. \\

In the end it was decided to have a Dashboard along each marker with some data
relevant to that marker, and some kind of selectable list which enables the
user to select the level to be displayed, all of Norway, each area, a segment,
or a stretch with each station. 

It was also concluded on having a second workshop where the content of the
dashboard should be decided. This resulted in \Ref{sub:workshop_2014_04_24}.

% subsection workshop_2014_04_04 (end)

\clearpage
\subsection{Workshop 2014-04-24} % (fold)
\label{sub:workshop_2014_04_24}
The second workshop was meant to determine what kind of statistical data was 
to be implemented in the dashboard mentioned in \Ref{sub:workshop_2014_04_04}.
Attending this workshop was Andreas Amdal Seim (SINTEF), Andreas Dypvik 
Landmark (SINTEF), Rimmert van der Kooij (SINTEF), and Magnus Krane (NTNU).\\

The workshop started with a brainstorming for different things to implement in
the map. The ideas were then ranged on how practical it is from 1-3 where 1 is
unpractical and 3 is very practical; and ranged on how desirable it is from 1-3
where 1 is undesirable and 3 is very desirable. The result ended up as \Ref{table:dashboard_functionality_wants_vs_needs}.

\begin{table}[!h]\small
	\begin{tabularx}{\textwidth}{|X|l|l|l|}
		\hline
		Functionality & Practicability & Desirable & Priority\\
		\hline
		Cause & 3 & 2 & 6\\
		\hline
		Suspensions & 3 & 1 & 3\\
		\hline
		Traffic density & 3 & 3 & 9\\
		\hline
	 	Worst stretch/station/train number & 3 & 2 & 9\\
		\hline
		Slow driving & 3 & 3 & 9\\
		\hline
		Delays & 3 & 3 & 9\\
		\hline
		Variation & 1 & 3 & 3\\
		\hline
		Outstanding errors & 1 & 1 & 1\\
		\hline
		Season effects & 3 & 1 & 3\\
		\hline
		Crossings & 3 & 3 & 9\\
		\hline
		Follow delays & 1 & 3 & 3\\
		\hline
		Speed limits & 3 & 1 & 3\\
		\hline
	\end{tabularx}
\caption{Dashboard functionality brainstorming ideas}
\label{table:dashboard_functionality_wants_vs_needs}
\end{table}

Based on this, it was selected to implement only the options which was ranged
as a 3 both on practicability and desirability:

\begin{itemize}
  \item Delays
  \item Speed restrictions
  \item Crossings
  \item Traffic density
\end{itemize}

The second goal of the workshop was to determine how the selected ideas was to
be implemented in both the map. It was desirable to both display some data in
the dashboard besides each marker, and display top 5 lists. These lists was
decided to display for delays and slow drivings. 

\subsubsection{Traffic density} % (fold)
\label{ssub:traffic_density}
The first idea chosen to display was traffic density. The issue to be displayed
is the effect the volume of trains driven throughout the day has on delays and
slow driving, and vice versa.

Several options to display traffic density were discussed, the problem was how
to correctly show the number of trains based on what kind of data is 
available. Should you display each train based on the train numbers and 
calculate for the entire line both ways? Should you display train numbers 
divided by the segment directors? 

In the workshop it was decided to show number of trains that passes each block
segment, since this is the data that is available. When selecting different 
levels upwards in the hierarchy, the number of train will be aggregated upon.
For instance, when on stretch level, the system will display each block 
between stations; on line level, the system will display an average of the 
stretches belonging to that line.

% subsubsection traffic_density (end)

\subsubsection{Crossings} % (fold)
\label{ssub:crossings}
The second idea was crossings between trains, how could delays and slow 
driving affect crossings. 

One of the problems discussed here was should you into account the difference
between actual crossings and planed crossings or not? Is it is hard to
calculate the difference because one does not know whether one of the trains
involved in the crossing has been canceled or other hard to document
occurrences, it was decided to only show a number for both actual crossings 
and planned crossings. These numbers will aggregate when changing a level 
upwards, in similar manner as the traffic density. 
% subsubsection crossings (end)

\subsubsection{Delays} % (fold)
\label{ssub:delays}
Since delays can cover a large area depending on how you define it, this had to
be properly discussed and defined. Should you follow the Norwegian standard
\cite{jernbaneverketPunklighetsTall} of only say that a train is delayed at the
final destination or use data from the signal post on the stretches compared to
the schedule? Should you take the variation into account?

It was decided to display the sum of the seconds the of delays, and sum of 
trains arriving to early. These sums would then be displayed in the dashboard 
and be aggregated according to the level.

% subsubsection delays (end)

\subsubsection{Speed restrictions} % (fold)
\label{ssub:speed_restrictions}
Visualization of slow driving has been a goal along the way since the beginning
and was natural to include from here on as well.

In the dashboard it was decided to show the top 5 upper and lower bounds.
It was also decided to include a little marker on the map to indicate the
location of the incident.

% subsubsection speed_restrictions (end)

% subsection workshop_2014_04_245 (end)

% section workshops (end)
