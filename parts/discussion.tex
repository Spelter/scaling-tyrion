% !TEX root=../thesis.tex
\chapter{Discussion}
\label{chapter:discussion}

\section{Stakeholders and aggregation} % (fold)
\label{sec:discussion_stakeholders_and_aggregation}
It is important to define the needs and requirements of the stakeholders early 
in the process, in order to design a system that gives the right scope. By the 
aid of workshops we defined the requirements, as it is part of the research 
method (see \Ref{sec:workshops}). We can define the stakeholders in a 
hierarchy, based on their organizational structure, and responsibility areas, 
this helps us to define the requirements to order the structure of the 
information.

There is a good synergy between the requirements and the system, when the 
system is designed with the requirements in mind. The synergy between the 
requirements and the system both have advantages and disadvantages. On one 
side the system has a great possibility to satisfy the requirements and needs 
of the stakeholders, since the system is tightly connected to the 
requirements. On the other side, the system can be inflexible to large 
changes, if the requirements change late in the project. In the workshops, we 
defined that the stakeholders must have the ability to view different kinds of 
information; the system must take these needs into consideration. A system 
aware of the stakeholders needs might end up being more complex, and has to 
process a larger amount of data in order to give the system the required 
functionality. Processing larger amounts of data leads to a more complex back-
end system. The system can be more dynamic, although complex, by reusing parts 
of the data processing for different information types, when designed with the 
needs as well as the requirements in mind. In order to meet the requirements 
of the stakeholders needs we have to aggregate the data. There are several 
ways to aggregate the groups of data according to the given criteria. 
Discussion can be around the method we used for the aggregation of the data, 
which enables the best view for the stakeholders. For the best view we take 
into account for example the following alternatives: average, minimum, 
maximum, median, and count. To decide the aggregation method, we have to 
consider how each information type is structured, but also how this structure 
fits in with the stakeholder their hierarchy. During the workshops, we decided 
that the best way to aggregate the data is to take the average, the decision 
was based on how the data is connected to the stakeholders and their
hierarchy.

When returning the processed data to the user, the system can use the 
requirements to aggregate the presentation of the data. Since the stakeholder 
their hierarchy contains the responsibility area of each stakeholder, the 
system can fit the visual view to the responsibility area of the current 
stakeholder. The system uses processed data for aggregating the hierarchy to 
find the stakeholder. Fitting the view to the stakeholder provides detail for 
a single stakeholder, this is useful for presenting a single stakeholder and 
their responsibility area. Since the system is aware of the stakeholders,
aggregating for a stakeholder and fitting the view to a stakeholder, can lead
to a conflict of interest between stakeholders and occurs due to overlapping 
responsibility areas. The conflict is extremely difficult to consider when
presenting the data and is therefor not taken into consideration. We decided 
to fit the visual presentation of data to the current stakeholder, and have 
the opportunity to jump, back and forth, within the hierarchy of the
stakeholders. By fitting the view to the stakeholder, the view enabled the 
user to receive more relevant information on the current stakeholder.

As the area director wants to study data from the entire area for a large 
period of time, and the segment director wants to study almost every detail 
that occurs on the segment, manual navigating in time was decided to use as a 
way to increase and decrease the level of detail in the data presented.


Keywords:

\begin{itemize}
	\item Important to define requirements early, in order to make an appropriate system, that is useful for the stakeholders
	\item How to define the stakeholders and requirements for use with a system?
	\begin{itemize}
		\item important to define who are the stakeholders
		\item important to define what the stakeholders needs are
		\item important to define to which need are connected to which stakeholder, in order to show the correct information in the system connected to each of the stakeholders needs.
	\end{itemize}
	\item How can we design a system that used the requirements needed by the stakeholders to process the data and visualize the data in such a way that the stakeholders needs are met?
	In order to meet the requirements of the stakeholders needs we need to aggregate the data. The following .....
	\item aggregation of data:
	\begin{itemize}
		\item two goals: geographic \& time
		\item aggregation based on detail level of visualization
		\begin{itemize}
			\item based on stakeholders need
			\begin{itemize}
				\item detail level in map
				\item which area is shown in the map
				\item aggregation based on time
			\end{itemize}
		\end{itemize}
	\end{itemize}
	\item Data aggregation methods
	\begin{itemize}
		\item What are good ways to aggregate the data?
		\item How to determine what fits the requirements?
	\end{itemize}
	\item How to present the data to the users based on the requirements? (detail
	level per stakeholder in the hierarchy)
	\item Time navigation for zooming through the data.
\end{itemize}


% section stakeholders_and_aggregation (end)

\section{Data storage and data queries} % (fold)
\label{sec:discussion_data_storage_and_data_queries}
By having several types of information which have different structure and
originating from different sources, the question of how the data storage should
be structured gets raised. One option for the structure is to merge all the
data sets into one large database. When the data is stored in one large
database
 When having one large, merged 
database, one experiences little maintainability, which is the degree where a
product or system can be modified by the intended maintainers\cite[p. 195]{Bass:2012:SAP:2392670}. 
One large database also means that we avoid duplication of data.
Another way to structure the data storage is to store the data sets in
independent storages with a loose coupling between the structures. Since most 
changes to a system happens after initial release, as described by Bass, 
Clements, and Kazman \cite[pp. 117-124]{Bass:2012:SAP:2392670},
means that when the data sets are designed with built-in flexibility, 
exercising the built-inn flexibility is usually cheaper than to hand-
code a specific change in hindsight. By having these loosely linked data sets,
one have achieved a modifiable data storage. This thesis demonstrated this
principal by having the stations stored in a hierarchy data set based one the 
stakeholder hierarchy, and the other data sets connected to the stations.\\



Keywords:
\begin{itemize}
	\item Data set structure in prototype
	\begin{itemize}
		\item How should the data sets be structured when the sets are based 
		on the stakeholders need for different type of information?
		\begin{itemize}
			\item Connection to the requirements of the stakeholders.
			\begin{itemize}
				\item Stakeholder responsibility area from the hierarchy
			\end{itemize}
			\item Advantages and disadvantages with one or several sets in storage.
			\begin{itemize}
				\item Scalability and modifiability ref quality attributes \cite{Bass:2012:SAP:2392670}
				\item Duplication of data with several sets of data.
			\end{itemize}
		\end{itemize}
	\end{itemize}
	\item Data queries
	\begin{itemize}
		\item Limit queries for data processing efficiency 
		and more responsive systems
		\begin{itemize}
			\item By the aid of time navigation and requirements, limit the 
			queries for more efficient calls to the data storage.
			\item The system will be more efficient in the processing of the 
			data, when limiting the quires.
		\end{itemize}
		\item Front end vs back end data processing for system efficiency and
		limiting data traffic
		\item Should the data be processed in the front-end or the back-end 
		of the system?
		\begin{itemize}
			\item The data should be aggregated in the back-end of the system, 
			since it is more efficient.
			\begin{itemize}
				\item Quicker calls to the database from the back-end.
				\item Less data traffic during transfer of.
				\item Usually quick hardware on server side, while it have a 
				great span in the front-end.
			\end{itemize}
		\end{itemize}
	\end{itemize}
\end{itemize}
% section data_storage_and_queries (end)

\section{Review of own work} % (fold)
\label{sec:review_of_own_work}
%Keywords:
%\begin{itemize}
%	\item Appropriate method?
%	\item 
%\end{itemize}
% section review_of_own_work (end)
