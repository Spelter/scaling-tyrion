% !TEX root=../thesis.tex
\chapter{Discussion}
\label{chapter:discussion}

\section{Stakeholders and aggregation} % (fold)
\label{sec:discussion_stakeholders_and_aggregation}

When one is supposed to have a system which is aware of the needs and
requirements of the stakeholder, we found it important to have clearly 
defined the requirements and needs of the stakeholders. By defining these
requirements and needs with the help of workshops(see \Ref{sec:workshops}), we 
started to build a prototype around them. As part of these requirements, a 
hierarchy of the stakeholders was defined, which were used to implement the 
prototype.  

By having a system which is so dependent on the hierarchy, it means that on 
one side one has a strong connection to the requirements and therefor is 
able to be aware of the stakeholder. On the other side it means that the 
system is not very flexible in terms of redefining the requirements. As the
workshops also defined that the stakeholders have the need for the ability to 
view different kind of information, the system must take these needs into
consideration also. Since the system now has more information to process, one
could argue that this affects how the system is aware of the requirements 
since this increases the complexity of the system. However, since this was 
defined along side the requirements, by having internal variables for this in 
the same manner as the stakeholders; the system is to use these needs for the
benefit of the stakeholder. \\

Since the requirements states the stakeholders are defined in a hierarchy with
larger and larger responsibility area upwards in the hierarchy, we implemented
aggregation as a way to processes the data according to the stakeholders. When
aggregation, there are several different ways to group together the different
values according to a given criteria. Between the different aggregation method,
such as average, minimum and maximum, median, and count, it is possible to
argue which fits the stakeholders the best. When deciding for the aggregation
method, not only does one have to consider how each information type is
structured, but also how this structure fits against the stakeholder hierarchy.
Since the data is connected to the stakeholders by the responsibility areas, 
it was concluded in the workshops that the aggregation method that fitted best
with the requirements was to aggregate through the data.

As important as the data processing is, one must not forget the presentation of
the data. By using the hierarchy based list of areas of responsibility for the
stakeholders, one has the opportunity to fit the visual display to the current
stakeholder. In one way, this can prove useful since it only displays
information relevant for the current stakeholder, in another way it can be more
useful to present more overall information. By having the opportunity to jump
back and forth within the hierarchy of stakeholders, we decided to fit the visual
presentation of data to the current stakeholder. Since the display was adapted to the user, it enabled the user to receive more information on the current
stakeholder.



Keywords:
\begin{itemize}
	\item Stakeholder
	\begin{itemize}
		\item How does the system utilizes the hierarchy of stakeholders for awareness?
		\item How does the system fit the requirements and needs?
	\end{itemize}
	\item Aggregating
	\begin{itemize}
		\item How to aggregate according to the given hierarchy
		\item How does the aggregation and hierarchy affect the visual presentation.
	\end{itemize}
\end{itemize}

% section stakeholders_and_aggregation (end)

\section{Data storage and data queries} % (fold)
\label{sec:discussion_data_storage_and_data_queries}
Keywords:
\begin{itemize}
	\item Data set structure in prototype
	\begin{itemize}
		\item Stakeholder responsibility consideration
		\item One or more sets
		\item Scalability and modifiability
	\end{itemize}
	\item Data queries
	\begin{itemize}
		\item Limit queries for data processing
		\item Front end vs back end data processing
		\item Data traffic
	\end{itemize}
\end{itemize}
% section data_storage_and_queries (end)

\section{Review of own work} % (fold)
\label{sec:review_of_own_work}
%Keywords:
%\begin{itemize}
%	\item Appropriate method?
%	\item 
%\end{itemize}
% section review_of_own_work (end)
