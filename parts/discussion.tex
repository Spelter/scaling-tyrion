% !TEX root=../thesis.tex
\chapter{Discussion}
\label{chapter:discussion}

\section{Stakeholders and aggregation} % (fold)
\label{sec:discussion_stakeholders_and_aggregation}

When one is supposed to have a system which is aware of the needs and
requirements of the stakeholder, we found it important to have clearly 
defined the requirements and needs of the stakeholders. By defining these
requirements and needs with the help of workshops(see \Ref{sec:workshops}), we 
started to build a prototype around them. As part of these requirements, a 
hierarchy of the stakeholders was defined, which were used to implement the 
prototype.  

By having a system which is so dependent on the hierarchy, one side means that 
one has a strong connection to the requirements and therefor is 
able to be aware of the stakeholder. The other side however, means that the 
system is not very flexible in terms of redefining the requirements. As the
workshops also defined that the stakeholders have the need for the ability to 
view different kind of information, the system must take these needs into
consideration also. Since the system now has more information to process, one
could argue that the amount of information to process affects how the system 
is aware of the requirements since the amount of information increases the 
complexity of the system. However, since the need for different type of 
information was defined along side the requirements, by having internal 
variables for the information type in the same manner as the stakeholders; the 
system is to use these needs for the benefit of the stakeholder. \\

Since the requirements states the stakeholders are defined in a hierarchy with
larger and larger responsibility area upwards in the hierarchy, we implemented
aggregation as a way to processes the data according to the stakeholders. When
aggregation, there are several different ways to group together the different
values according to a given criteria. Between the different aggregation method,
such as average, minimum and maximum, median, and count, one could argue that 
there is possible to argue which methods fits the stakeholders the best. When 
deciding for the aggregation method, not only does one have to consider how 
each information type is structured, but also how this structure fits against 
the stakeholder hierarchy. Since the data is connected to the stakeholders by 
the responsibility areas, the decision was made with the help of the workshops 
that the aggregation method that fitted best with the requirements was to 
average through the data.

As important as the data processing is, one must not forget the presentation of
the data. By using the hierarchy based list of areas of responsibility for the
stakeholders, one has the opportunity to fit the visual display to the current
stakeholder. In one way, to fit the display to the stakeholder can prove 
useful since the presentation only displays information relevant for the 
current stakeholder, in another way to let the user decide what level of 
visualization the visual display shall present can be more since the display 
then have the opportunity to present more overall information. By having the 
opportunity to jump
back and forth within the hierarchy of stakeholders, we decided to fit the 
visual presentation of data to the current stakeholder. Since the display was 
adapted to the user, the display enabled the user to receive more relevant 
information on the current stakeholder.

% section stakeholders_and_aggregation (end)

\section{Data storage and data queries} % (fold)
\label{sec:discussion_data_storage_and_data_queries}
By having several types of information which have different structure and
originating from different sources, raises the question of how the data storage
shall be structured. One way to structure the data storage, is to merge all the
data sets into one large set and store that. When having one large, merged 
data set, one experiences little maintainability, which is the degree where a
product or system can be modified by the intended maintainers\cite[p. 195]{Bass:2012:SAP:2392670}.
Another way to structure the data storage is to store the data sets in
independent storages with a loose coupling between the structures. Since most 
changes to a system happens after initial release, as described by Bass, 
Clements, and Kazman \cite[pp. 117-124]{Bass:2012:SAP:2392670},
means that when the data sets are designed with built-in flexibility, 
exercising the built-inn flexibility is usually cheaper than to hand-
code a specific change in hindsight.By having these loosely linked data sets,
one have achieved a modifiable data storage. This thesis demonstrated this
principal by having the stations stored in a hierarchy data set based one the 
stakeholder hierarchy, and the other data sets connected to the stations.\\



Keywords:
\begin{itemize}
	\item Data set structure in prototype
	\begin{itemize}
		\item Stakeholder responsibility consideration
		\item One or more sets
		\item Scalability and modifiability
	\end{itemize}
	\item Data queries
	\begin{itemize}
		\item Limit queries for data processing
		\item Front end vs back end data processing
		\item Data traffic
	\end{itemize}
\end{itemize}
% section data_storage_and_queries (end)

\section{Review of own work} % (fold)
\label{sec:review_of_own_work}
%Keywords:
%\begin{itemize}
%	\item Appropriate method?
%	\item 
%\end{itemize}
% section review_of_own_work (end)
