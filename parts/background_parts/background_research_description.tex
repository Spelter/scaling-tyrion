% !TEX root=../../thesis.tex
\clearpage
\section{Stakeholders} % (fold)
\label{sec:back_stakeholders}
When looking at the Norwegian railway, there exists several type of users which
may be taken into consideration. On the Norwegian railway operates several
companies, who all have several positions in its organization hierarchy. Each
of these positions have different responsibilities and therefor different
interests in the system. A example of such a hierarchy is presented in
\Ref{table:user_roles}.

\begin{table}[!h]\small
	\begin{tabularx}{\textwidth}{|l|X|}
		\hline
		User type & Responsibility \\
		\hline
		Railway director & Organization director\\
		\hline
		Infrastructure director & Responsibility to facilitate that the railway traffic in Norway is safe and reliable\\
		\hline
		Area director & Responsibility to coordinate traffic within a area\\
		\hline
		Stretch director & Responsibility to coordinate traffic within certain
		stretches\\
		\hline
		Segment director & Responsibility to coordinate traffic within a segment\\
		\hline
	\end{tabularx}
\caption{User roles in the Norwegian railway network}
\label{table:user_roles}
\end{table}

This lists only relevant users from Jernbaneverket (see Section
\Vref{sub:subsection_jernbaneverket}), where the organization map is presented
online\cite{jernbaneverketOrganisasjon}\cite{jernbaneverketInfrastruktdivisjon}.
Many of the other companies which operates on the Norwegian railway system,
have a similar structure. 

This leads to many users which have different needs
in what kind of information that will be presented. A segment director may need
to know every single detail which happens on the segment he is responsible for,
while a stretch director may need to only know the major details from every
segment that is in his stretch. 

% section stakeholders (end)

\clearpage
\section{Aggregation} % (fold)
\label{sec:back_aggregation}
The responsibility hierarchy based on the users presented in Section \Ref{sec:back_stakeholders} have 
defined the way data have been presented in this thesis. Since the higher up in the
hierarchy one may look, a even larger set of data is available. When presenting
a large set of data, one may aggregate over this set. This means that groups of
data are replaced with summary statistics based on those data\cite{wikiAggregation}. 

% section aggregation (end)

\clearpage
\section{Data Sets} % (fold)
\label{sec:back_data_sets}

\begin{itemize}
	\item Large data sets
	\item Individual data sets
\end{itemize}
% section data_sets (end)
