% !TEX root=../../thesis.tex

\section{Defining frameworks} % (fold)
\label{sec:defining_frameworks}
In this section we will first define the users in the railway network, and then
define framework(s) based on the need of the users and the information the
systems in \Ref{sect:backgroundExamples} presents.

% subsection information_plotted_vs_information_needed (end)
\subsection{Functionality description} % (fold)
\label{sub:functionality_description}
In this section we define different types of information and determines the
system which covers each type.

\begin{description}
	\item [National]	Whether or not it gives a national overview.
	\item [Route]		Whether or not it is possible to study a whole route.
	\item [Stretch]		Whether or not it is possible to study individual 						stretches.
	\item [Live]		Whether or not it gives a live view of the current 						situation.
	\item [Historical]	Whether or not it is possible to study historical data.
\end{description}

\begin{table}[!h]\small
	\begin{tabularx}{\textwidth}{|p{4.7cm}|c|c|c|c|c|}
		\hline
		\diaghead(-4,1){\hskip4.7cmm}%
			{System}{Functionality}
		 	& National & Route & Stretch & Live &
			Historical\\
		\hline
		\Ref{fig:zugmonitor} \nameref{fig:zugmonitor} & X & - & - & X & X\\
		\hline
		\Ref{fig:ukLiveMap} \nameref{fig:ukLiveMap} & X & - & - & - & X\\
		\hline
		\Ref{fig:muniLightRail} \nameref{fig:muniLightRail} & - & X & X & X & - \\
		\hline
		\Ref{fig:miserymap} \nameref{fig:miserymap} & X & - & - & - & X \\
		\hline
		\Ref{fig:jernbaneverket-punklighet} \nameref{fig:jernbaneverket-punklighet} & X & - & - &
		- & X\\
		\hline
		\Ref{fig:jernbaneverket-tios} \nameref{fig:jernbaneverket-tios} & - & X & X &
		- & X\\
		\hline
		\Ref{fig:taag-info-kart} \nameref{fig:taag-info-kart} & X & - & - & X & -\\
		\hline
		\Ref{fig:taag-info-historik} \nameref{fig:taag-info-historik} & X & X & X & X
		& X\\
		\hline
		\Ref{fig:krysningsinteraksjon} \nameref{fig:krysningsinteraksjon} & - & - & X &
		- & X \\
		\hline
		\Ref{fig:live-punklighet} \nameref{fig:live-punklighet} & - & X & X & - & X\\
		\hline
		\Ref{fig:plot-spc-for-strekning} \nameref{fig:plot-spc-for-strekning} & - & - & X &
		- & X\\
		\hline
		\Ref{fig:plot-spc-for-stasjonsopphold} \nameref{fig:plot-spc-for-stasjonsopphold} & - & - & X
		& - & X \\
		\hline
		\Ref{fig:ukespunklighet} \nameref{fig:ukespunklighet} & - & X & X & - & X \\
		\hline
		\Ref{fig:cargonet} \nameref{fig:cargonet} & X & - & X & X & -\\
		\hline
	\end{tabularx}
\caption{Functionality description}
\label{table:functionality_description}
\end{table}

% subsection functionality_description (end)
\clearpage
\subsection{Type of information presented} % (fold)
\label{sub:information_presented}
In this section we summaries what kind of information the different systems presents. 

\begin{table}[!h]\small
	\begin{tabularx}{\textwidth}{|l|X|}
		\hline
		System & Type of information\\
		\hline
		\Ref{fig:zugmonitor} \nameref{fig:zugmonitor} & Delays \\
		\hline
		\Ref{fig:ukLiveMap} \nameref{fig:ukLiveMap} & Delays \\
		\hline
		\Ref{fig:muniLightRail} \nameref{fig:muniLightRail} & Delays and schedule \\
		\hline
		\Ref{fig:miserymap} \nameref{fig:miserymap} & Delays \\
		\hline
		\Ref{fig:jernbaneverket-punklighet} \nameref{fig:jernbaneverket-punklighet} & Punctuality \\
		\hline
		\Ref{fig:jernbaneverket-tios} \nameref{fig:jernbaneverket-tios} & Delays and schedule \\
		\hline
		\Ref{fig:taag-info-kart} \nameref{fig:taag-info-kart} & Delays  \\
		\hline
		\Ref{fig:taag-info-historik} \nameref{fig:taag-info-historik} & Delays \\
		\hline
		\Ref{fig:krysningsinteraksjon} \nameref{fig:krysningsinteraksjon} & Delays and crossings \\
		\hline
		\Ref{fig:live-punklighet} \nameref{fig:live-punklighet} & Delays \\
		\hline
		\Ref{fig:plot-spc-for-strekning} \nameref{fig:plot-spc-for-strekning} & Time used vs others \\
		\hline
		\Ref{fig:plot-spc-for-stasjonsopphold} \nameref{fig:plot-spc-for-stasjonsopphold} & Time used vs others \\
		\hline
		\Ref{fig:ukespunklighet} \nameref{fig:ukespunklighet} & Punctuality \\
		\hline
		\Ref{fig:cargonet} \nameref{fig:cargonet} & Delays \\
		\hline
	\end{tabularx}
\caption{Information presented}
\label{table:information_presented}
\end{table}

As seen in \Ref{table:information_presented}, almost all of the system found 
presents either delays or punctuality in some way. The comparison shows that 
there are gaps in what kind of information that is being displayed.

% subsection information_presented (end)

\subsection{User needs vs system information} % (fold)
\label{sub:information_presented_vs_information_needed}

Since each project presented in \Vref{sect:backgroundExamples} is
based on different types of users accessing the systems, they present 
different amount of information. Since some of the projects presented contains 
different types of information presentations and they may cover different 
needs, the following definition will be based on the presented figures. 

\begin{table}[!h]\small
	\begin{tabularx}{\textwidth}{|p{1.4cm}|p{1.4cm}|X|}
		\hline
		User & Need & Figure \\
		\hline
		Nation director & Overview for the nation & 
				\begin{tabular}{p{3.7cm}p{3.7cm}}
						\Ref{fig:zugmonitor} \nameref{fig:zugmonitor}, &
						\Ref{fig:ukLiveMap} \nameref{fig:ukLiveMap}, \\
						\Ref{fig:miserymap} \nameref{fig:miserymap}, &
						\Ref{fig:taag-info-kart} \nameref{fig:taag-info-kart}, \\
						\Ref{fig:taag-info-historik} \nameref{fig:taag-info-historik}, &
						\Ref{fig:live-punklighet} \nameref{fig:live-punklighet}, \\
						\Ref{fig:cargonet} \nameref{fig:cargonet} & \\
				\end{tabular} \\
		\hline
		Area director & Overview / detailed for the area & 
				\begin{tabular}{p{3.7cm}p{3.7cm}}
						\Ref{fig:zugmonitor} \nameref{fig:zugmonitor}, &
						\Ref{fig:ukLiveMap} \nameref{fig:ukLiveMap}, \\
						\Ref{fig:muniLightRail} \nameref{fig:muniLightRail}, &
						\Ref{fig:jernbaneverket-tios} \nameref{fig:jernbaneverket-tios}, \\
						\Ref{fig:krysningsinteraksjon} \nameref{fig:krysningsinteraksjon}, &
						\Ref{fig:plot-spc-for-strekning} \nameref{fig:plot-spc-for-strekning}, \\
						\Ref{fig:plot-spc-for-stasjonsopphold} \nameref{fig:plot-spc-for-stasjonsopphold}, &
						\Ref{fig:ukespunklighet} \nameref{fig:ukespunklighet}\\
				\end{tabular} \\
		\hline
		Segment director & Detailed for each segment & 
				\begin{tabular}{p{3.7cm}p{3.7cm}}
						\Ref{fig:muniLightRail} \nameref{fig:muniLightRail}, &
						\Ref{fig:jernbaneverket-tios} \nameref{fig:jernbaneverket-tios}, \\
						\Ref{fig:krysningsinteraksjon} \nameref{fig:krysningsinteraksjon}, &
						\Ref{fig:plot-spc-for-strekning} \nameref{fig:plot-spc-for-strekning}, \\
						\Ref{fig:plot-spc-for-stasjonsopphold} \nameref{fig:plot-spc-for-stasjonsopphold}, &
						\Ref{fig:ukespunklighet} \nameref{fig:ukespunklighet}\\
				\end{tabular} \\
		\hline
		Traffic responsible & Detailed for each line & 
				\begin{tabular}{p{3.7cm}p{3.7cm}}
						\Ref{fig:zugmonitor} \nameref{fig:zugmonitor}, &
						\Ref{fig:ukLiveMap} \nameref{fig:ukLiveMap}, \\
						\Ref{fig:miserymap} \nameref{fig:miserymap}, &
						\Ref{fig:taag-info-kart} \nameref{fig:taag-info-kart}, \\
						\Ref{fig:live-punklighet} \nameref{fig:live-punklighet}, &
						\Ref{fig:ukespunklighet} \nameref{fig:ukespunklighet}, \\
						\Ref{fig:cargonet} \nameref{fig:cargonet} & \\
				\end{tabular} \\
		\hline
		East responsible & Detailed for east & 
				\begin{tabular}{p{3.7cm}p{3.7cm}}
						\Ref{fig:muniLightRail} \nameref{fig:muniLightRail}, &
						\Ref{fig:jernbaneverket-tios} \nameref{fig:jernbaneverket-tios}, \\
						\Ref{fig:krysningsinteraksjon} \nameref{fig:krysningsinteraksjon}, &
						\Ref{fig:plot-spc-for-strekning} \nameref{fig:plot-spc-for-strekning}, \\
						\Ref{fig:plot-spc-for-stasjonsopphold} \nameref{fig:plot-spc-for-stasjonsopphold}, &
						\Ref{fig:ukespunklighet} \nameref{fig:ukespunklighet}\\
				\end{tabular} \\
		\hline
		None & Basic & 	
				\begin{tabular}{p{3.7cm}p{3.7cm}}
						\Ref{fig:zugmonitor} \nameref{fig:zugmonitor}, &
						\Ref{fig:ukLiveMap} \nameref{fig:ukLiveMap}, \\
						\Ref{fig:miserymap} \nameref{fig:miserymap}, &
						\Ref{fig:jernbaneverket-punklighet} \nameref{fig:live-punklighet}\\
				\end{tabular} \\
		\hline
	\end{tabularx}
\caption{User needs vs system information}
\label{table:information_presented_vs_information_needed}
\end{table}

% section defining_frameworks (end)
