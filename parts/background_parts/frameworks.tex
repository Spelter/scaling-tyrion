% !TEX root=../../thesis.tex

\section{Defining frameworks} % (fold)
\label{sec:defining_frameworks}
This section will first define the users in the railway network, and then
define framework(s) based on the need of the users and what the examples
presented in section \vref{sect:backgroundExamples} tries to present.

\subsection{User roles in the Norwegian railway network} % (fold)
\label{sub:user_roles_in_the_norwegian_railway_network}

Combined in all the companies with a certain interest in the Norwegian railway 
network there are several types of users which have different perspective of
the railway. 
\begin{table}[!h]\small
	\begin{tabularx}{\textwidth}{|l|l|X|}
		\hline
		Company & User type & Responsibility \\
		\hline
		Jernbaneverket & Traffic director & Responsibility to facilitate that the railway traffic in Norway is safe and reliable\\
		\hline
		Jernbaneverket & Area director & Responsibility to facilitate that the railway traffic in Norway is safe and reliable\\
		\hline
		Jernbaneverket & Segment director & Responsibility to facilitate that the railway traffic in Norway is safe and reliable\\
		\hline
		NSB & Nation responsible & 1 per line (Dovre, Bergen, etc.)\\
		\hline
		NSB & East responsible & 1 per defined area in the east of Norway (Vestfold county, Oslo - Hamar,
		etc.)\\
		\hline
		None & Average Joe & General user interested in train punctuality\\
		\hline
	\end{tabularx}
\caption{User roles in the Norwegian railway network}
\label{table:user_roles}
\end{table}

Based on the perspective of these users, different information is useful and
not all examples in section \ref{sect:backgroundExamples} may be useful. They
may have just a basic interest whether trains are delayed or not, for instance
traveling with trains on holiday; or have a more detailed need to understand
all train delays in the whole network, for instance to make new schedules. 

% subsection user_roles_in_the_norwegian_railway_network (end)

\clearpage
% subsection information_plotted_vs_information_needed (end)
\subsection{Functionality description} % (fold)
\label{sub:functionality_description}
This section tries to present some functionalities and in a tabular tries to
present which examples cover which functionality. 
\begin{description}
	\item [National]	Whether or not it gives a national overview.
	\item [Route]		Whether or not it is possible to study a whole route.
	\item [Stretch]		Whether or not it is possible to study individual 						stretches.
	\item [Live]		Whether or not it gives a live view of the current 						situation.
	\item [Historical]	Whether or not it is possible to study historical data.
\end{description}

\begin{table}[!h]\small
	\begin{tabularx}{\textwidth}{|X|c|c|c|c|c|}
		\hline
		\backslashbox[87mm]{System}{Function} & National & Route & Stretch & Live &
		Historical\\
		\hline
		\Ref{fig:zugmonitor} \nameref{fig:zugmonitor} & X & - & - & X & X\\
		\hline
		\Ref{fig:ukLiveMap} \nameref{fig:ukLiveMap} & X & - & - & - & X\\
		\hline
		\Ref{fig:muniLightRail} \nameref{fig:muniLightRail} & - & X & X & X & - \\
		\hline
		\Ref{fig:miserymap} \nameref{fig:miserymap} & X & - & - & - & X \\
		\hline
		\Ref{fig:jernbaneverket-punklighet} \nameref{fig:jernbaneverket-punklighet} & X & - & - &
		- & X\\
		\hline
		\Ref{fig:jernbaneverket-tios} \nameref{fig:jernbaneverket-tios} & - & X & X &
		- & X\\
		\hline
		\Ref{fig:taag-info-kart} \nameref{fig:taag-info-kart} & X & - & - & X & -\\
		\hline
		\Ref{fig:taag-info-historik} \nameref{fig:taag-info-historik} & X & X & X & X
		& -\\
		\hline
		\Ref{fig:krysningsinteraksjon} \nameref{fig:krysningsinteraksjon} & - & - & X &
		- & X \\
		\hline
		\Ref{fig:live-punklighet} \nameref{fig:live-punklighet} & - & X & X & - & X\\
		\hline
		\Ref{fig:plot-spc-for-strekning} \nameref{fig:plot-spc-for-strekning} & - & - & X &
		- & X\\
		\hline
		\Ref{fig:plot-spc-for-stasjonsopphold} \nameref{fig:plot-spc-for-stasjonsopphold} & - & - & X
		& - & X \\
		\hline
		\Ref{fig:ukespunklighet} \nameref{fig:ukespunklighet} & - & X & X & - & X \\
		\hline
		\Ref{fig:cargonet} \nameref{fig:cargonet} & X & - & X & X & -\\
		\hline
	\end{tabularx}
\caption{Functionality description}
\label{table:functionality_description}
\end{table}

% subsection functionality_description (end)


\clearpage
\subsection{Information presented vs information needed} % (fold)
\label{sub:information_presented_vs_information_needed}

Since each project presented in section \vref{sect:backgroundExamples} is
based different type of user accessing it, they present different amount of
information. Since some the projects presented contains different types of
information presentations and they may cover different needs, the following 
definition will be based on the presented figures. 

\begin{table}[!h]\small
	\begin{tabularx}{\textwidth}{|p{1.8cm}|p{2.5cm}|X|}
		\hline
		User & Need & Figure \\
		\hline
		Traffic director & Overview for the nation & 
				\begin{tabular}{p{4.8cm}l}
						\Ref{fig:zugmonitor} \nameref{fig:zugmonitor}, &
						\Ref{fig:ukLiveMap} \nameref{fig:ukLiveMap}, \\
						\Ref{fig:miserymap} \nameref{fig:miserymap}, &
						\Ref{fig:taag-info-kart} \nameref{fig:taag-info-kart}, \\
						\Ref{fig:taag-info-historik} \nameref{fig:taag-info-historik}, &
						\Ref{fig:live-punklighet} \nameref{fig:live-punklighet}, \\
						\Ref{fig:cargonet} \nameref{fig:cargonet} & \\
				\end{tabular} \\
		\hline
		Area director & Overview / detailed for the area & 
				\begin{tabular}{p{4.8cm}l}
						\Ref{fig:zugmonitor} \nameref{fig:zugmonitor}, &
						\Ref{fig:ukLiveMap} \nameref{fig:ukLiveMap}, \\
						\Ref{fig:muniLightRail} \nameref{fig:muniLightRail}, &
						\Ref{fig:jernbaneverket-tios} \nameref{fig:jernbaneverket-tios}, \\
						\Ref{fig:krysningsinteraksjon} \nameref{fig:krysningsinteraksjon}, &
						\Ref{fig:plot-spc-for-strekning} \nameref{fig:plot-spc-for-strekning}, \\
						\Ref{fig:plot-spc-for-stasjonsopphold} \nameref{fig:plot-spc-for-stasjonsopphold}, &
						\Ref{fig:ukespunklighet} \nameref{fig:ukespunklighet}\\
				\end{tabular} \\
		\hline
		Segment director & Detailed for each segment & 
				\begin{tabular}{p{4.8cm}l}
						\Ref{fig:muniLightRail} \nameref{fig:muniLightRail}, &
						\Ref{fig:jernbaneverket-tios} \nameref{fig:jernbaneverket-tios}, \\
						\Ref{fig:krysningsinteraksjon} \nameref{fig:krysningsinteraksjon}, &
						\Ref{fig:plot-spc-for-strekning} \nameref{fig:plot-spc-for-strekning}, \\
						\Ref{fig:plot-spc-for-stasjonsopphold} \nameref{fig:plot-spc-for-stasjonsopphold}, &
						\Ref{fig:ukespunklighet} \nameref{fig:ukespunklighet}\\
				\end{tabular} \\
		\hline
		Nation responsible & Detailed for each line & 
				\begin{tabular}{p{4.8cm}l}
						\Ref{fig:zugmonitor} \nameref{fig:zugmonitor}, &
						\Ref{fig:ukLiveMap} \nameref{fig:ukLiveMap}, \\
						\Ref{fig:miserymap} \nameref{fig:miserymap}, &
						\Ref{fig:taag-info-kart} \nameref{fig:taag-info-kart}, \\
						\Ref{fig:live-punklighet} \nameref{fig:live-punklighet}, &
						\Ref{fig:ukespunklighet} \nameref{fig:ukespunklighet}, \\
						\Ref{fig:cargonet} \nameref{fig:cargonet} & \\
				\end{tabular} \\
		\hline
		East responsible & Detailed for east & 
				\begin{tabular}{p{4.8cm}l}
						\Ref{fig:muniLightRail} \nameref{fig:muniLightRail}, &
						\Ref{fig:jernbaneverket-tios} \nameref{fig:jernbaneverket-tios}, \\
						\Ref{fig:krysningsinteraksjon} \nameref{fig:krysningsinteraksjon}, &
						\Ref{fig:plot-spc-for-strekning} \nameref{fig:plot-spc-for-strekning}, \\
						\Ref{fig:plot-spc-for-stasjonsopphold} \nameref{fig:plot-spc-for-stasjonsopphold}, &
						\Ref{fig:ukespunklighet} \nameref{fig:ukespunklighet}\\
				\end{tabular} \\
		\hline
		None & Basic & 	
				\begin{tabular}{p{4.8cm}l}
						\Ref{fig:zugmonitor} \nameref{fig:zugmonitor}, &
						\Ref{fig:ukLiveMap} \nameref{fig:ukLiveMap}, \\
						\Ref{fig:miserymap} \nameref{fig:miserymap}, &
						\Ref{fig:jernbaneverket-punklighet} \nameref{fig:live-punklighet}\\
				\end{tabular} \\
		\hline
	\end{tabularx}
\caption{Information presented vs information needed}
\label{table:information_presented_vs_information_needed}
\end{table}

% section defining_frameworks (end)
