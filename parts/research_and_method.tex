% !TEX root=../thesis.tex

\chapter{Research questions and method} % (fold)
\label{cha:research_questions_and_method}

\section{Workshops} % (fold)
\label{sec:workshops}
In order to determine what is relevant content for this thesis, workshops were
held. These workshops were meant to bring some context and direction for the
thesis. 

\subsection{Workshop 2014-04-04} % (fold)
\label{sub:workshop_2014_04_04}
The first workshop was meant to bring some clarity to what kind of users this
system will expect to be used by; their responsibility, and their needs.
It was also meant to how the system will relate to these users and their needs.
Attending this workshop was Andreas Amdal Seim (SINTEF), Andreas Dypvik 
Landmark (SINTEF), Rimmert van der Kooij (SINTEF), Nils Olsson (NTNU), Per 
Magnus Hegglund (Jernbaneverket), Magnus Bae (NTNU), and Magnus Krane (NTNU).

% subsection workshop_2014_04_04 (end)

\subsection{Workshop 2014-04-24} % (fold)
\label{sub:workshop_2014_04_24}
The second workshop was meant to determine what kind of statistical data was 
to be implemented in the dashboard mentioned in \ref{sub:workshop_2014_04_04}.
Attending this workshop was Andreas Amdal Seim (SINTEF), Andreas Dypvik 
Landmark (SINTEF), Rimmert van der Kooij (SINTEF), and Magnus Krane (NTNU).\\

The workshop started with a brainstorming for different things to implement in
the map. The ideas were then ranged on how practical it is from 1-3 where 1 is
unpractical and 3 is very practical; and ranged on how desirable it is from 1-3
where 1 is undesirable and 3 is very desirable. The result ended up as \ref{table:dashboard_functionality_wants_vs_needs}.

\begin{table}[!h]\small
	\begin{tabularx}{\textwidth}{|X|l|l|}
		\hline
		Functionality & Practicability & Desirable \\
		\hline
		Cause & 3 & 2\\
		\hline
		Suspensions & 3 & 1\\
		\hline
		Traffic density & 3 & 3\\
		\hline
	 	Worst stretch/station/train number & 3 & 2\\
		\hline
		Slow driving & 3 & 3\\
		\hline
		Delays & 3 & 3\\
		\hline
		Variation & 1 & 3\\
		\hline
		Outstanding errors & 1 & 1\\
		\hline
		Season effects & 3 & 1\\
		\hline
		Crossings & 3 & 3\\
		\hline
		Follow delays & 1 & 3\\
		\hline
		Speed limits & 3 & 1\\
		\hline
	\end{tabularx}
\caption{Dashboard functionality brainstorming ideas}
\label{table:dashboard_functionality_wants_vs_needs}
\end{table}

Based on this, it was selected to implement only the options which was ranged
as a 3 both on practicability and desirability:

\begin{itemize}
  \item Delays
  \item Slow driving
  \item Crossings
  \item Traffic density
\end{itemize}

The second goal of the workshop was to determine how the selected ideas was to
be implemented in both the map. It was desirable to both display some data in
the dashboard besides each marker, and display some top 5 lists for worsts
incidents.

\clearpage
\subsubsection{Traffic density} % (fold)
\label{ssub:traffic_density}
The first idea chosen to display was traffic density. The issue to be displayed
is the effect the volume of trains driven throughout the day has on delays and
slow driving, and vice versa.

Several options to display traffic density was discussed, the problem was how
to correctly show the number of trains based on what data is available. Should
you display each train based on the train numbers and calculate for the entire
line both way? Should you display train numbers divided by the segment
directors? 

In the workshop it was decided to show number of trains that passes each block
segment, which is what we have data on. This will aggregate upwards when
selecting different levels to display. For instance, when on stretch level, it
will display each block between stations; on line level, it will display an
average of the stretches belonging to that line.

% subsubsection traffic_density (end)

\subsubsection{Crossings} % (fold)
\label{ssub:crossings}
The second idea was crossings between trains, how could delays and slow 
driving affect crossings. 

One of the problems discussed here was should you into account the difference
between actual crossings and planed crossings or not? Since this is hard to
calculate because of the difference does not take into account whether or not
one of the trains involved in the crossing has been canceled and other hard to
document occurrences such as that, it was decided to only show a number for
both actual crossings and planned crossings. These numbers will aggregate when
changing level upwards in similar manners as the traffic density. 
% subsubsection crossings (end)

\subsubsection{Delays} % (fold)
\label{ssub:delays}
Since delays can cover a large area depending on how you define it, this had to
be properly discussed and defined. Should you follow the Norwegian standard
\cite{jernbaneverketPunklighetsTall} of only say that a train is delayed at the
final destination or use data from the signal post on the stretches compared to
the schedule? Should you take the variation into account?

It was decided to display the sum seconds of delays, and sum of trains
arriving to early. These sums would then be displayed in the dashboard and be
aggregated accordingly to level.

% subsubsection delays (end)

\subsubsection{Slow driving} % (fold)
\label{ssub:slow_driving}

% subsubsection slow_driving (end)

% subsection workshop_2014_04_245 (end)

% section workshops (end
% chapter research_questions_and_method (end)

